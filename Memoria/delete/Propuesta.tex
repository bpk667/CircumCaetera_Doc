\section{Identificación de la Propuesta}
El presente documento recoge la propuesta asociada al Proyecto de Fin de Grado {\em \titulo} presentada por
{\em \autor} cuyo historial de versiones se detalla a continuación:
\subsection{Control de versiones del documento}
\begin{center}
  \begin{tabularx}{0.8\textwidth} {llll}
		\multicolumn{1}{c}{\textbf{Versión}} &
		\multicolumn{1}{c}{\textbf{Fecha}} &
		\multicolumn{1}{c}{\textbf{Autor}} &
		\multicolumn{1}{c}{\textbf{Descripción}}\\
		\hline
    1.0 & 30/12/2018  & Pedro Pozuelo & Primera versión de la propuesta \\
  \end{tabularx}
\end{center}


\section{Identificación del Proyecto}
% \pfg{Título y acrónimo}
% \pfg{resumen de un párrafo de extensión, del objetivo prioritario del proyecto}
% \pfg{nombre, temática y objetivo principal}
\par El título del proyecto propuesto es: {\em \titulo}.
\par El objetivo del proyecto es crear una solución de software libre de firewall de aplicación web perimetral con
capacidades de aceleración TLS que permita proteger aplicaciones web con in\-de\-pen\-den\-cia de su arquitectura
para lo que actuará como un proxy inverso sobre una tecnología de contenedores de software tipo Docker.


\section{Intervinientes principales en el proyecto}
% \pfg{equipo desarrollador}
% \pfg{profesor que desempeña el papel de director del proyecto}
% \pfg{cliente (el director-cliente).}
% \pfg{número identificador del PFG}

\par Equipo desarrollador: \autor
\par Director de proyecto: \profesor
\par Número de identificador del proyecto: \PFGID

\clearpage
\section{Resumen del proyecto}
% \pfg{mayor detalle que el párrafo informativo incluido en el apartado 2 de la Propuesta (entre media cara y una cara de DIN-A4.)}
% \pfg{El resumen deberá incluir el objetivo global del proyecto, aunque sin detallar ni de\-sa\-rro\-llar.}
% \pfg{Tras la lectura de este apartado, el lector debe quedar informado del problema que afronta el proyecto y del alcance de la solución.}

\subsection{Versión en español}
\par El objetivo del proyecto es construir una solución de software libre con capacidades de firewall de aplicación web (en adelante WAF, de sus siglas en inglés, {\em Web Application Firewall}) y aceleración SSL/TLS.
\par Actualmente la mayoría de los ataques se realizan contra aplicaciones web, con lo que es cada vez más importante
contar con una solución que sea capaz de analizar el tráfico web y proteger las aplicaciones.
\par Por otro lado, en los últimos años existe una tendencia a publicar los servicios web sobre canales cifrados y el
tráfico sin cifrar es cada vez menor. Este cambio tiene un impacto en el rendimiento de las plataformas sobre los que
se ejecutan los servicios web y añade complejidad. Máxime cuando recientemente se han descubierto múltiples
vulnerabilidades en los protocolos SSL/TLS que requieren realizar continuamente cambios y actualizaciones.
\par Actualmente existen soluciones propietarias que ofrecen funcionalidades WAF y aceleración SSL / TLS, pero son muy
costosas y sólo son viables en proyectos con suficiente envergadura y presupuesto, quedando fuera del alcance en
aplicaciones web con menos presupuesto o que no generen suficientes beneficios para justificar su inversión.
\par Por otro lado, existen soluciones WAF de software libre, que tradicionalmente funcionan como módulos adicionales
al servidor web, como por ejemplo {\em modSecurity} como módulo del servidor web {\em Apache}. Este tipo de
soluciones requiere por lo tanto un ejercicio de integración con las aplicaciones web y consumen recursos del
servidor que pueden impactar en el rendimiento.
\par Adicionalmente, para la implementación y configuración adecuada de estos módulos se requiere de una figura con
conocimientos de seguridad, y si se despliegan dentro del servicio web, se requiere que la figura responsable de la
plataforma web configure unos componentes para los que carece de los conocimientos necesarios y se requiere que asuma
el rol de administrador de servicios que desconoce.
\par Es por ello que se propone una solución que funcione en su propio contenedor o servidor, lo
que permitirá desplegarla de manera independiente a la plataforma, con lo que no impactará a los recursos de la
arquitectura web, y permitirá una administración basada en roles y que los cambios realizados en uno de los
componentes no afecten a otros componentes.

\subsection{English version}
\par The goal of the project is to build a free software solution with WAF (Web Application Firewall) capabilities and SSL/TLS acceleration.
\par Nowadays, most of the attacks are run against web applications, hence it is more important than ever to have a
mechanism able to analyze web traffic and protect web applications.
\par On the other hand, during the past few years it is more common to publish web services over encrypted channels
instead of traditional decrypted ones. This trend impacts servers' performance where the web services are running
and adds complexity. Especially, since recently several vul\-ne\-ra\-bi\-li\-ties in the SSL/TLS protocols have been published,
requiring configuration changes and applying updates.
\par There are proprietary solutions that give us the WAF and SSL/TLS acceleration capabilities, but they are costly
and they are only affordable for projects with enough magnitude and budget. So it is not worth it to deploy these type
of solutions when the web applications don't have enough budget in order to justify the investment.
\par There are also free software WAFs, which run as a web server module, for instance {\em modSecurity} is a WAF
module for Apache. These solutions require to be integrated as part of the web applications and they consume server
resources that can impact on server's performance.
\par Additionally, the deployment and setup of these modules require security knowledge and, if they are deployed
within the web server, it'd mean the person responsible for web administration, who may not have the proper knowledge,
would need to set up the security components and would need to assume a role for tasks he/she is not qualified.
\par For all the reasons previously stated, I propose an autonomous solution running in its own con\-tai\-ner or server,
which will be platform independent and will not impact on the web platform resources. It will also allow an
administration based on roles (RBAC) and any configuration change would only affect its own component.

\begin{comment}
\end{comment}

\section{Antecedentes y estado actual del tema}
% \pfg{características del problema a abordar, las técnicas empleadas en el pasado para su solución y los resultados obtenidos. También deberá investigarse cómo se está abordando el pro\-ble\-ma actualmente y con qué resultados.}
% \pfg{listado comentado de la bibliografía más relevante.}

\par Dentro de las soluciones de seguridad tradicionales, nos encontramos los firewall de red pe\-ri\-me\-tra\-les que
permiten proteger los servicios que no se quieren publicar en Internet, pero estas soluciones no permiten analizar o
proteger la capa de aplicación de los servicios web que sí se publican, y permiten todo el tráfico dirigido a los
servicios web, sea este legítimo o una amenaza de seguridad.

\par Para proteger estos servicios web, una de las claves es mejorar los patrones de desarrollo incluyendo principios
y buenas prácticas de desarrollo seguro, pero estas medidas no son suficientes por diversas causas: se descubren
nuevos ataques que no se conocían en el momento de realizar el desarrollo, los equipos de desarrollo no están
adecuadamente formados, no existen los controles de validación adecuados como parte del ciclo de desarrollo, existen
aplicaciones en producción que no se actualizan cuando se descubre una nueva vulnerabilidad, etc.
\par Por estos y otros motivos no se puede considerar que las aplicaciones web sean seguras y se deben desplegar
controles de seguridad que ayuden a protegerlas frente a los ataques.
\par Las soluciones WAF nos permiten analizar este tipo de tráfico y proteger las aplicaciones web.
\par Dentro de las soluciones WAF, existen los siguientes tipos: Soluciones de tipo appliance, soluciones en la nube
de tipo Software as a Service (en adelante SaaS) como parte de servicios de Red de dis\-tri\-bu\-ción de contenidos (en adelante CDN, de sus siglas en inglés, {\em Content Delivery Network}) y soluciones de tipo software.
\par Las soluciones de tipo appliance o CDN son propietarias y muy costosas, por lo que sólo son viables en proyectos
donde se justifique la inversión.
\par Las soluciones de tipo software requieren que se contemplen como parte del diseño de la aplicación web, por lo
que requieren un ejercicio de integración con el servicio web y consumen recursos del servidor y puede afectar al
rendimiento.
\par En materia de aceleración SSL/TLS, nos encontramos que en los últimos 5 años se ha producido un cambio
significativo; el mercado ha apostado por utilizar canales cifrados HTTPS de forma masiva frente a la política previa
en la que sólo se cifraban ciertas comunicaciones que se consideraban sensibles.
\par Paralelamente a este cambio, se han publicado múltiples ataques a los protocolos SSL y TLS que han supuesto que
la práctica tradicional de habilitar cifrado por defecto y no cambiarlo ya no sea válida. Actualmente son habituales
los cambios de configuración de los {\em ciphersuites}, certificados y configuración en general de los protocolos.
Estos cambios no son triviales y deben realizarse sobre los terminadores del protocolo que están expuestos a
Internet.
\par Es por ello que se propone una solución que pueda desplegarse y configurarse de forma in\-de\-pen\-dien\-te a la
plataforma web, con lo que se asegure que un cambio en el componente de seguridad no afectará a la plataforma web y
viceversa.


\subsection{Bibliografía}
\begin{itemize}
  \item OWASP Top 10 Most Critical Web Application Security Risks:\\
    \href{https://www.owasp.org/index.php/Category:OWASP\_Top\_Ten\_Project}
    {https://www.owasp.org/index.php/Category:OWASP\_Top\_Ten\_Project}
  \item Modelo de amenazas SSL propuesto por {\em Qualys}:\\
    \href{https://www.ssllabs.com/projects/ssl-threat-model/index.html}
    {https://www.ssllabs.com/projects/ssl-threat-model/index.html}
  \item Majority of the world’s top million websites now use HTTPS:\\
    \href{https://www.welivesecurity.com/2018/09/03/majority-worlds-top-websites-https/}
    {https://www.welivesecurity.com/2018/09/03/majority-worlds-top-websites-https/}
  \item HTTPS encryption on the web:\\
    \href{https://transparencyreport.google.com/https/overview?hl=en}
    {https://transparencyreport.google.com/https/overview?hl=en}
  \item ModSecurity:\\
    \href{http://www.modsecurity.org/}
    {http://www.modsecurity.org/}
  \item Fabricante de appliances WAF lider del mercado:\\
    \href{https://www.imperva.com/products/web-application-firewall-waf/}
    {https://www.imperva.com/products/web-application-firewall-waf/}
\end{itemize}


\section{Objetivos}
\subsection{Objetivo global}
% \pfg{objetivo global, que es explicado con un lenguaje general y sin entrar en gran detalle.}
\par El objetivo del proyecto es crear una solución de software libre de firewall de aplicación web perimetral con
capacidades de aceleración TLS que permita proteger aplicaciones web con in\-de\-pen\-den\-cia de su arquitectura
para lo que actuará como un proxy inverso sobre una tecnología de contenedores de software tipo Docker.

\subsection{Objetivos concretos}
\par Tengo algunos objetivos identificados pero no están detallados (a la espera de reunirme con el profesor
asignado).
\begin{comment}
    \pfg{objetivos concretos. múltiples, y deben necesariamente presentarse numerados.}
    \pfg{\ul{objetivos concretos deben describirse con la corrección suficiente como para que los miembros del tribunal examinador puedan objetivamente de\-ter\-mi\-nar o me\-dir su gra\-do de cum\-pli\-mien\-to.}}
    \pfg{verificables por terceras personas}
    \begin{enumerate}
      \item \shit
      \item \shit
      \item \shit
    \end{enumerate}

    \pfg{Deberá quedar claro, en este apartado, la relación entre los objetivos concretos y el objetivo global, mediante
    la explicación de cómo la consecución de los primeros consigue la con\-se\-cu\-ción del segundo.}
    \shit

    \pfg{Deberá especificarse si el tema está amparado por el Plan Nacional de Investigación Científica,
    Desarrollo e Innovación Tecnológica vigente. En caso afirmativo, deberá especificarse en qué
    Programa Nacional y en qué apartado.}
    \shit
\end{comment}

\subsection{Objetivos no contemplados}
\par Pendiente de definir. A la espera de reunirme con el profesor asignado.

\section{Compromisos y requisitos}
\subsection{Compromisos del cliente}
\par Pendiente de definir. A la espera de reunirme con el profesor asignado.

\subsection{Requisitos del usuario}
\par Pendiente de definir. A la espera de reunirme con el profesor asignado.

\subsection{Requisitos del sistema}
% \pfg{Especifica los requisitos que el sistema deberá cumplir exigidos por los usuarios.}
% \pfg{Deberán especificarse las prestaciones y requisitos mínimos del sistema informático que se requerirá para soportar la solución informática que proporciona el proyecto ajustada a los márgenes de rendimiento que puedan exigirse.}

\par La solución se construirá sobre un sistema operativo Debian GNU/Linux, el cual a su vez podrá ser desplegado
sobre una plataforma de contenedores de software tipo Docker o instancias de la nube como AWS o Azure.
\par Se requeriría que las entradas DNS de las aplicaciones web puedan apuntar al servicio WAF o bien se modifique el
enrutamiento de red de forma que el WAF éste en un punto de la red externo a la aplicación web.
\par La solución no implementará gestión de certificados o gestionará la arquitectura web que protege.

\section{Plan de trabajo y objetivos del proyecto}
\subsection{Plan de trabajo}
\par Pendiente de definir. A la espera de reunirme con el profesor asignado.
\par Se seguirán las guías y buenas prácticas promovidas por el {\em Center for Internet Security} y {\em OWASP}
\begin{comment}
    \pfg{desglosarse en tareas y subtareas. Estas irán numeradas
    de una manera indicativa de su jerarquía (por ejemplo: 7, 7.1, 7.1.1, etc.). Para cada tarea o
    subtarea se especificará: interviniente responsable, otros participantes, breve descripción,
    consumo de recursos previsto, dependencias con otras tareas, duración y plazo de ejecución.}
    \pfg{Además se incluirá un cronograma o diagrama de Gantt. Deberán figurar hitos, entre los que se
    considerarán:
    \begin{itemize}
      \item Disponibilidad de versiones preliminares o de prueba.
      \item Disponibilidad de documentación entregable.
      \item Reuniones de evaluación.
    \end{itemize}
    }

    \pfg{Descripción de todo lo que vas a hacer, tecnología a emplear, modelos, patrones, a\-pli\-ca\-cio\-nes anteriores, etc., que centren la idea del trabajo futuro.}

    \pfg{detallar qué vas a hacer y qué metodologías y/o técnicas aplicarás en cada etapa del Proyecto}

    \pfg{Puedes aplicar los principios del proceso software.}

    \pfg{Puedes aplicar metodologías (de desarrollo software), normas (de calidad, etc.) técnicas asociadas al desarrollo de tecnologías, aplicaciones, etc.}
    \begin{itemize}
      \item OWASP
      \item CIS
    \end{itemize}
\end{comment}

\subsection{Presupuesto del proyecto}
\par No se contempla que sea necesario un presupuesto económico.
\par Pendiente de confirmar con el profesor asignado.

\section{Beneficios para el cliente}
% \pfg{cómo se beneficiaría el cliente de la realización del proyecto.}
% \pfg{La Propuesta debe reflejar el aporte que el estudiante pretende dar a su Proyecto, basado en líneas de trabajo ya existentes pero aportando "algo extra" (más eficiencia, combinación de técnicas para mejorar los resultados, etc.).}
% \pfg{Lo que propones cubre una necesidad o mejora algo existente.}

\par El protocolo HTTP es usado y atacado masivamente. Las soluciones actuales son o bien de pago y cerradas -
tradicionalmente un modelo de appliance o CDN) no protegen adecuadamente las a\-pli\-ca\-cio\-nes debido a la complejidad de
integrarlas como parte de la aplicación web o tienen una alta complejidad de desplegar y mantener.
\par La solución propuesta es gratuita, software libre y se despliega en su propio contenedor o servidor, lo
que permitirá ahorrar costes, adaptarla a las necesidades del cliente y desplegarla de manera in\-de\-pen\-dien\-te a la
plataforma web.

\section{Experiencia previa en el tema}
% \pfg{el cliente valore la capacidad del equipo del proyecto y de la organización proveedora para desarrollar exitosamente el proyecto}
\par Desde hace más de 10 años he trabajado en múltiples proyectos en los que se he desplegado y administrado
diversas tecnologías propietarias WAF, destacando las soluciones tipo appliances, como por ejemplo Secure Sphere
Imperva, y soluciones CDN, como Akamai Kona.
\par Antes de dedicarme a la seguridad era administrador de sistemas y siempre he sido un entusista del software
libre y un gran defensor de sus bondades.
\par Por lo tanto tengo experiencia en las tecnologías WAF privativas tradicionales así como en las plataformas
Linux y creo que este proyecto puede cubrir una carencia que actualmente existe.

\section{Viabilidad y plan de recursos}
\subsection{Estudio de viabilidad técnica}
% \pfg{estudio que determine si técnicamente el proyecto de sistema informático que se propone es viable con la tecnología actual. El cliente empleará este estudio para cuantificar los riesgos que corre al aceptar la Propuesta, y evaluar su componente de innovación.}
% \pfg{abordando aspectos concretos del ámbito de trabajo seleccionado y orientado a ga\-ran\-ti\-zar el buen término del Proyecto}
\par Pendiente de definir. A la espera de reunirme con el profesor asignado.

\subsection{Plan de recursos}
% \pfg{esbozo preliminar. En dicho plan deberán reflejarse qué recursos se consumirán en cada etapa del proyecto, y cuál es su coste estimado}
\par El proyecto se realizará individualmente.

\subsection{Estudio de viabilidad económica}
% \pfg{pequeño estudio sobre si los recursos económicos contemplados son suficientes para garantizar el desarrollo del proyecto}
\par No se contempla que el proyecto requiera una inversión económica. Pendiente de confirmar con el profesor asignado.

\section{Comentarios}
% \pfg{clarificar puntos que deban ser enfatizados acerca del proyecto. También es útil para clarificar aspectos que se prevea que puedan ser motivo de controversia a lo largo del desarrollo del proyecto (por ejemplo, la manera en la que una variación de los requisitos influirá en un sobrecoste del proyecto).}

\par Conozco bien las soluciones privativas del mercado y, si bien tecnológicamente son soluciones potentes, no
pueden ser modificadas y carecen de la adaptabilidad del software libre.
\par Por otro lado, estas soluciones tienen unos precios muy elevados, lo que hace que no sean viables para
aplicaciones web que no generan unos beneficios significados; y, por lo tanto, tradicionalmente estas aplicaciones no
son protegidas adecuadamente.
\par Con la solución propuesta el objetivo es crear un WAF con capacidad de aceleración SSL/TLS que permita proteger
aplicaciones web sin requerir una inversión significativa y sin añadir complejidad a la arquitectura web existente.

\section{Aceptación del proyecto}
\par Pendiente de aceptación.


\end{document}

%%%%%%%%%%%%%%%%%%%%%%%%%%%%%%%%%%%%%%%%%%%%%%%%%%
\clearpage
\section{Unstructured}
\subsection{Introducción}
\subsection{Situación actual}
\subsection{Propuesta}
\shit

\subsubsection{Necesidad que cubre}
\shit

\subsubsection{Objetivo de la propuesta}
\shit

\subsubsection{Mejoras que aporta la solución propuesta}

\subsubsection{Principios del proceso de software}
\shit

\subsubsection{Metodologías, normas, técnicas}
\shit

\subsubsection{Plan de trabajo}
\shit

\subsection{Estudio de viabilidad}
\shit

\subsection{Descripción de la propuesta}
\shit


%%%%%%%%%%%%%%%%%%%%%%%%%%%%%%%%%%%%%%%%%%%%%%%%%%


\section{Garbage}
\par En una primera fase se evaluaron distintos objetivos a negociar y finalmente se eligió negociar una modalidad de tele-trabajo en un piloto a nivel departamental.
\par Si bien otros países en la empresa cuentan con esa posibilidad, actualmente no es posible en el área al que pertenezco, por lo que consideré que este trabajo
era una buena oportunidad para cambiar la situación y lanzar un piloto.
\par Una vez identificado el objetivo inicial, procedí a recabar información interna de la empresa con el fin de sopesar la viabilidad de la petición. Puse especial
foco en documentos oficiales internos de la empresa en los que se referenciase la posibilidad del tele-trabajo o se le diese importancia a la conciliación de la
vida laboral y familiar.
\par Afortunadamente, no sólo encontré documentos en los que se mencionaba la posibilidad de tele-trabajo sino que existía una propuesta de iniciativa que no se
había puesto en marcha.
\par Con estos documentos identificados, me puse en contacto con el responsable de departamento, José Luis Jerez y le comenté en líneas generales la existencia de
esta práctica y el objetivo que había elegido.
\par Aunque su reacción fue positiva y ha estado dispuesto a colaborar en el proceso de negociación, lo cierto es que no se ha comprometido a llevar a cabo la
propuesta después de que termine la actividad. No obstante, mi intención es continuar adelante con el piloto y espero poder utilizar el momento generado
por este trabajo para llevar el piloto a buen puerto.

\section{Estrategia}
\par A la hora de plantear la estrategia de negociación, se eligió un acercamiento ganar-ganar, pues se considera que este tipo de negociaciones tienen más posibilidades de llegar a buen puerto si ambas partes salen beneficiadas.
\subsection{Planificación táctica}
\par Se optó por convocar una reunión inicial de 30 minutos en una sala de reuniones con el fin de que se le pudiese dar una introducción adecuada a la persona convocada y no hubiese interrupciones.
\par Una vez se consiguió que la persona convocada accediese a participar en la negociación (podríamos considerar que es la negociación de la negociación), se convocó una segunda reunión en la que se produjo la negociación en sí.
\par La convocatoria de esta negociación se realizó mediante el siguiente correo:
\begin{displayquote}
  {\em 
\par Buenos días, José Luis.
\par Antes de nada, gracias por tu apoyo y por el tiempo dedicado a las reuniones de la práctica de la universidad. Sé que estamos todos liados y realmente te
  agradezco tu ayuda.
\par Tal como quedamos en la reunión de introducción, querría volver a convocarte para que podamos definir cómo podríamos lanzar un piloto de tele-trabajo y qué necesitaríamos para ponerlo en marcha.
\par Inicialmente te convoco para el hueco de 10 a 11 del lunes, que aparece libre en tu calendario. Si ves que no te viene bien, no dudes en proponer otra hora que
  te venga mejor (yo tengo hueco el lunes por la tarde a partir de las 4).
\par El orden del día de la reunión será el siguiente:
  \begin{itemize}
    \item Evaluar ventajas, desventajas, riesgos y beneficios del tele-trabajo para la empresa y para el empleado.
    \item Definir alcance del piloto. Identificar un conjunto de usuarios que puedan participar.
    \item Identificar qué requisitos debe cumplir la empresa para poder ofrecer tele-trabajo.
    \item Identificar qué requisitos son necesarios para que un trabajador pueda optar al tele-trabajo.
    \item Identificar fases y actividades asociadas.
\end{itemize}
\par Las personas convocadas a la reunión son:
  \begin{itemize}
    \item José Luis Jerez como responsable del departamento en el que se quiere lanzar el piloto.
    \item Pedro Pozuelo como solicitante de la iniciativa y alumno de la Universidad Europea.
  \end{itemize}
  \par En cuanto me confirmes tu disponibilidad reservo una sala.

  \par Muchas gracias de nuevo.
  \par Un saludo,
  \par Pedro.}
\end{displayquote}
\subsection{Tácticas de negociación empleadas}
\subsubsection{Usar documentos impresos}
\par A la hora de preparar la negociación, se utilizó la táctica de usar documentos impresos con el fin de darle un mayor peso a la petición.
\par Como parte de la cultura empresarial existe una política sobre la conciliación de la vida laboral y familiar y hay múltiples menciones en esta dirección en documentos internos dirigidos al trabajador. Con el fin de fortalecer los argumentos a favor de la propuesta se hizo una labor de investigación y se imprimieron los documentos más relevantes antes de asistir a la reunión de negociación.
\subsubsection{Empezar lejos del BATNA}
\par Otra táctica que se utilizó fue empezar lejos del BATNA. En el planteamiento inicial no se había mencionado el alcance - cuántos días a la semana o cuanta gente podía participar en el proyecto - y lo que se planteó al arranque de la reunión fue incluir a todo el departamento de ingeniería y tener varios días a la semana de tele-trabajo. A la hora de plantearlo se sabía que esta propuesta no iba a ser aceptada, pero me permitió iniciar una dinámica de negociaciones/concesiones que nos llevó a un resultado satisfactorio.

\section{Desarrollo de las negociaciones}
\par Tal como se ha comentado, el punto de arranque es muy ambicioso e implica un riesgo difícilmente asumible por la empresa. El motivo por el que se eligió este
acercamiento es que es una táctica que se utiliza frecuentemente entre elementos desiguales, en este caso la empresa y trabajador, donde suele haber un
tira-y-afloja y donde es generalizado que el resultado de una negociación sea distinto del punto de partida (por ejemplo, la experiencia que muchos hemos tenido en
la que el salario que pide un trabajador para entrar en una empresa no es el que finalmente consigue).
\par La estrategia consistió en arrancar la exposición con una recapitulación de cómo Roche promueve iniciativas similares como parte de su código ético empresarial, en este punto se presentaron los documentos oficiales de la empresa que previamente se habían preparado.
\par A continuación, se presentó una propuesta genérica en la que se mencionaba el tele-trabajo como algo abstracto que está alineado con otras iniciativas.
\par Después se realizó una primera petición muy generalista en la que el alcance y las condiciones eran muy ambiciosas y difícilmente alcanzables.
\par Partiendo de este punto, se procedió a identificar porqué dicho alcance no es viable y se fue ajustando el alcance hasta el punto final en el que ambas partes
quedaron satisfechas: El piloto durará 6 meses, permite un día a la semana de teletrabajo elegido por el trabajador y todos los trabajadores pertenecientes al
departamento de ingeniería que lo deseen pueden participar.

\section{Acuerdo y conclusión}
\par Finalmente se consiguió llegar a un acuerdo en el que ambas partes están satisfechas con el resultado (ver Anexo).
\par No obstante, como parte de la negociación tuve que realizar como concesión que el piloto sólo contemple un día a la semana. He de reconocer que hubiese preferido llegar a dos días a la semana, pero no fue posible y tuvimos que conformarnos con un día (eso sí, al menos el trabajador puede elegir qué día le viene mejor).
\par Por último, la mayor concesión que he tenido que realizar ha sido que la negociación no es vinculante en el entorno laboral (sólo a nivel de práctica de la
asignatura). Pero tampoco se ha descartado que se pueda seguir adelante con la iniciativa, con lo que es posible que se pueda poner en marcha. Creo que tendré que
seguir negociando.
