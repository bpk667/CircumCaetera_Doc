\chapter{Requisitos candidatos}
\par En esta sección se exponen los requisitos candidatos que se intentarán cubrir con nuestra solución.
\par En primer lugar se identifican los requisitos orientados principalmente al componente WAF:
\begin{enumerate}[\bfseries{}R1. ]
  \item La solución debe poder ejecutarse en un sistema operativo o máquina independiente de la plataforma de la aplicación web con el
    objetivo de garantizar independencia en las tareas de administración y permitir aplicar un modelo RBAC.
  \item La solución debe disponer de un conjunto básico de políticas de auditoría o bloqueo que permitan proteger la aplicación web frente a
    los ataques más comunes.
  \item La solución debe permitir la elaboración de reglas personalizadas según las necesidades específicas de la plataforma del cliente.
  \item La plataforma debe ser compatible con el modelo de licencias de {\em software libre\cite{softwarelibre}} tipo
    Licencia Pública General de GNU (en adelánte \acrshort{gpl}, de  sus siglas en inglés \acrlong{gpl}) o Licencia Pública General Reducida
    de GNU (en adelánte \acrshort{lgpl}, de  sus siglas en inglés \acrlong{lgpl}).
  \item La plataforma debe generar logs de seguridad exportables a soluciones externas de gestión de información y eventos de seguridad (en
    adelante \acrshort{siem} de sus siglas en inglés, \acrlong{siem}).
\end{enumerate}

\par A continuación se recogen los requisitos asociados al componente de TLS:
\begin{enumerate}[\bfseries{}R1. ]
  \item La solución debe poder participar en la negociación TLS, gestionando los certificados presentados a los clientes.
  \item Debe permitir aplicar soluciones de SSL offloading entre el WAf y los frontales de la plataforma web, en entornos controlados en los
    que prime el rendimiento, o establecer un segundo túnel TLS en las soluciones en las que el entorno tenga un mayor riesgo o prime la
    seguridad.
\end{enumerate}

% https://www.gnu.org/licenses/license-compatibility.html
