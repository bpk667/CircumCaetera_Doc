% \documentclass[a4paper,11pt,spanish]{article}
\documentclass[a4paper,11pt,spanish]{report}

\usepackage{amsmath, amssymb, latexsym}
\usepackage[margin=2.5cm]{geometry}
\usepackage{xltxtra}

% Highlight where is the paragraph overfull \hbox error (adding a black bar where a line is too wide)
\overfullrule=2cm

% Allows us to wrap underlined text
\usepackage{soul}

% \usepackage[spanish]{babel} % Se ha sustituido por Polyglossia
\usepackage{polyglossia}
  \setmainlanguage{spanish}
  \setmainfont[Mapping=tex-text]{Linux Libertine O}
  \defaultfontfeatures{Ligatures=TeX}

\usepackage{titlesec}
 \titleformat{\chapter}[display]
   {\normalfont\bfseries}{}{0pt}{\Huge}

\usepackage{ifpdf}
\usepackage{moreverb}
\usepackage{multicol}
\usepackage{epsfig}
\usepackage{tikz}
\usepackage{color}
\usepackage{colortbl}
\usepackage{enumerate}
\usepackage{titlesec}
\usepackage{appendix}
\addto\captionsspanish{%
  \appendixtitleon
  \renewcommand{\appendixname}{Anexo}
  \renewcommand\appendixpagename{Anexos} }
\usepackage{msc}                % Draw Message Sequence Charts
\usepackage{scalefnt}           % Rescale fonts to arbitrary sizes
\usepackage{xurl}
\usepackage[pdfencoding=unicode,
            unicode=true,
            bookmarks=true,
            colorlinks=true, 
            pdfstartview=FitH, 
            linkcolor=blue, 
            citecolor=blue, 
            breaklinks=true,
            bookmarksopen=true,
            urlcolor=blue]
            {hyperref}
\usepackage[all]{hypcap}
\usepackage[color={0 0 1}]{attachfile2}
\usepackage{listings}
\usepackage[all]{xy}            % Paquete para dibujar diagramas (\xymatrix)
\usepackage{changepage}         % Útil para modificar márgenes en mitad del documento
\usepackage{enumitem}
\usepackage{tabularx}
\usepackage{tabulary}
\usepackage{mathtools}
\usepackage{pdflscape}
\usepackage[autostyle]{csquotes} 

\selectlanguage{spanish}

%% Figures within a column...
\makeatletter
\newenvironment{tablehere}
{\def\@captype{table}}
{}
\newenvironment{figurehere}
{\def\@captype{figure}}
{}
\makeatother

\oddsidemargin 0in
\textwidth 6.25in

\title{\titulo{}}
\author{\autor{}}
\date{\today}

%% Información para el PDF que se genera.
\ifpdf
  \pdfinfo{
    /Author (\autor{})
    /title{\titulo{}}
  }
\fi

% Gestión de la bibliografía.
\usepackage[
  backend=biber,
  style=numeric,
  sorting=none
]{biblatex}

% Create glossary and definitions
\usepackage[acronyms,toc]{glossaries}

%% Información para los ficheros adjuntos.
\attachfilesetup{author=\autor{}}

%% Redefinimos el aspecto del título de las secciones
%\renewcommand{\thesection}{Problema \arabic {section}.}

%% Redefinimos el tamaño de la fuente en \section y \subsection
%\titleformat{\section}{\large\bfseries\em}{Parte \thesection.}{1em}{}
%\titleformat{\subsection}{\normalsize\bfseries}{Parte \thesubsection.}{1em}{}
%\titleformat{\subsubsection}{\normalsize\bfseries}{Parte \thesubsubsection.}{1em}{}

%% Definimos el tamaño de la fuente para verbatim en small
\makeatletter
  \g@addto@macro\@verbatim\small
\makeatother 

%% Indicamos un interlineado de 1.5
% \renewcommand*{\baselinestretch}{1.5}

%% Aumentamos el espacio de nuevo párrafo
\parskip=2mm

%% Propiedades del código fuente para todo el documento
\definecolor{dkgreen}{rgb}{0,0.6,0}
\definecolor{gray}{rgb}{0.5,0.5,0.5}
\definecolor{mauve}{rgb}{0.58,0,0.82}
\lstset{numbers=none,
        basicstyle=\scriptsize,
        frame=single,
        breaklines=true,
        %title=\lstname,
        language=sql,
        keywordstyle=\color{blue},
        commentstyle=\color{dkgreen},
        stringstyle=\color{mauve},
        extendedchars=true,
        texcl,
        tabsize=2}
\def\inline{\lstinline[language=sql,
		keywordstyle=\color{blue},
		commentstyle=\color{dkgreen},
		stringstyle=\color{mauve},
		extendedchars=true,
		texcl,
		basicstyle=\normalsize]}
