\begin{comment}
\subsubsection{Análisis de WAF de software libre}

\subsubsection{Conclusión}
\par Actualmente existen soluciones propietarias que ofrecen funcionalidades WAF y aceleración SSL / TLS, pero son muy
costosas y sólo son viables en proyectos con suficiente envergadura y presupuesto, quedando fuera del alcance en
aplicaciones web con menos presupuesto o que no generen suficientes beneficios para justificar su inversión.
\par Por otro lado, existen soluciones WAF de software libre, que tradicionalmente funcionan como módulos adicionales
al servidor web, como por ejemplo {\em modSecurity} como módulo del servidor web {\em Apache}. Este tipo de
soluciones requiere por lo tanto un ejercicio de integración con las aplicaciones web y consumen recursos del
servidor que pueden impactar en el rendimiento.

\par En materia de aceleración SSL/TLS, nos encontramos que en los últimos 5 años se ha producido un cambio
significativo; el mercado ha apostado por utilizar canales cifrados HTTPS de forma masiva frente a la política previa
en la que sólo se cifraban ciertas comunicaciones que se consideraban sensibles.
\par Paralelamente a este cambio, se han publicado múltiples ataques a los protocolos SSL y TLS que han supuesto que
la práctica tradicional de habilitar cifrado por defecto y no cambiarlo ya no sea válida. Actualmente son habituales
los cambios de configuración de los {\em ciphersuites}, certificados y configuración en general de los protocolos.
Estos cambios no son triviales y deben realizarse sobre los terminadores del protocolo que están expuestos a
Internet.
\par Es por ello que se propone una solución que pueda desplegarse y configurarse de forma in\-de\-pen\-dien\-te a la
plataforma web, con lo que se asegure que un cambio en el componente de seguridad no afectará a la plataforma web y
viceversa.


\section{Notas}

HTTP2
https://daniel.haxx.se/blog/tag/openssl/

Hardening:
Add "gpg --verify" to all the packages

  Compiling nginx with modsecuirty (security tips):
  "change the server: nginx name or do some thing like adding secure headers"
https://www.raspberrypi.org/forums/viewtopic.php?t=210247

https://sysadminnightmare.org/installing-modsecurity-with-nginx-and-owasp-crs-on-debian-stretch/

https://github.com/kjakub/docker-nginx-modsecurity-v3-waf/tree/master/waf

nginx -g daemon off;
nginx -t


\section{Compatibilidad licencias}
% https://www.gnu.org/licenses/license-compatibility.html


\end{comment}
