\begin{comment}
\subsubsection{Análisis de WAF de software libre}
\par Algunas soluciones se han descartado debido a que están discontinuadas desde hace tiempo: IronBee sin actualizar desde hace más de 3 años
y WebCastellum desde hace 5 años.
\par RAPTOR se ha descartado debido a que no soporta tráfico SSL/TLS y está en versión Beta.
\par NAXSI se ha considerado que no cumple los requisitos básicos que debe tener un WAF actualmente, pues sólo implementa dos controles de
seguridad - de los referenciados en el documento de buenas prácticas de OWASP\cite[apartado A3.2]{owaspbestpractices}); concretamente, sólo
protege ataques de tipo {\em Cross-site scripting} (en adelante \acrshort{xss}\cite{owaspxss}) e inyecciones SQL (en adelante
\acrshort{sqli}\cite{owaspsqli}). Además, se han producido duras críticas acerca de las funcionalidades que ofrece \cite{naxsianalisis}.
\par OpenWAF por su parte es una iniciativa con un planteamiento y unas funcionalidades muy interesantes, pero que lleva más de dos años sin
publicar una nueva versión y además carece de suficiente madurez para considerarse su uso en un entorno de producción (la última versión
publicada es la 0.0.4).
\par Respecto a FreeWAF (también conocido como {\em lua-resty-waf}) está en una situación muy similar a OpenWAF, pues su última versión tiene más de 2
años y se trata de la versión 0.11.1\cite{freewafchangelog}.
\par En el caso de Shadow Daemon, se trata de un conector de aplicación web con una funcionalidades interesantes, pero por diseño requiere de
su integración con el framework de desarrollo y sólo es compatible con PHP, Perl y Python. Por lo tanto no es posible implementarlo como
elemento externo e independiente de la plataforma web.
\par También se han descartado las siguientes soluciones debido a que no son compatibles con la plataforma GNU/Linux: WebKnight sólo es
compatible con el software de aplicación web Internet Information Services\cite{iis} y Vulture requiere un sistema operativo FreeBSD.


\subsubsection{Conclusión}
\par Actualmente existen soluciones propietarias que ofrecen funcionalidades WAF y aceleración SSL / TLS, pero son muy
costosas y sólo son viables en proyectos con suficiente envergadura y presupuesto, quedando fuera del alcance en
aplicaciones web con menos presupuesto o que no generen suficientes beneficios para justificar su inversión.
\par Por otro lado, existen soluciones WAF de software libre, que tradicionalmente funcionan como módulos adicionales
al servidor web, como por ejemplo {\em modSecurity} como módulo del servidor web {\em Apache}. Este tipo de
soluciones requiere por lo tanto un ejercicio de integración con las aplicaciones web y consumen recursos del
servidor que pueden impactar en el rendimiento.

\par En materia de aceleración SSL/TLS, nos encontramos que en los últimos 5 años se ha producido un cambio
significativo; el mercado ha apostado por utilizar canales cifrados HTTPS de forma masiva frente a la política previa
en la que sólo se cifraban ciertas comunicaciones que se consideraban sensibles.
\par Paralelamente a este cambio, se han publicado múltiples ataques a los protocolos SSL y TLS que han supuesto que
la práctica tradicional de habilitar cifrado por defecto y no cambiarlo ya no sea válida. Actualmente son habituales
los cambios de configuración de los {\em ciphersuites}, certificados y configuración en general de los protocolos.
Estos cambios no son triviales y deben realizarse sobre los terminadores del protocolo que están expuestos a
Internet.
\par Es por ello que se propone una solución que pueda desplegarse y configurarse de forma in\-de\-pen\-dien\-te a la
plataforma web, con lo que se asegure que un cambio en el componente de seguridad no afectará a la plataforma web y
viceversa.


\section{Notas}
Comparativa de backend SSL:
https://en.wikipedia.org/wiki/Comparison_of_TLS_implementations
GnuTLS        libcurl4-gnutls-dev
OpenSSL       libcurl4-openssl-dev
NSS           libcurl4-nss-dev



HTTP2
https://daniel.haxx.se/blog/tag/openssl/

Hardening:
Add "gpg --verify" to all the packages

  Compiling nginx with modsecuirty (security tips):
  "change the server: nginx name or do some thing like adding secure headers"
https://www.raspberrypi.org/forums/viewtopic.php?t=210247

https://sysadminnightmare.org/installing-modsecurity-with-nginx-and-owasp-crs-on-debian-stretch/

https://github.com/kjakub/docker-nginx-modsecurity-v3-waf/tree/master/waf

nginx -g daemon off;
nginx -t

error_log  logs/error.log  warn;



\end{comment}
