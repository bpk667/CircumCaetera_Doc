\section{Viabilidad Económica}
\par Uno de los principales motivos por los que las soluciones WAF no son más frecuentes, es su elevado valor económico, tal como se ha detallado en la sección de ~\nameref{subsec:estadoarte}.
\par Es por ello que se ha priorizado utilizar tecnologías sin coste de licenciamiento y minimizar otros gastos lo máximo posible.
\par No obstante, la solución tiene una serie de costes que se deben considerar:
\begin{itemize}
  \item Plataforma hardware o de Cloud sobre la que se desplegará la solución.
  \item Horas de técnico administrador de la plataforma.
  \item Soporte técnico del proveedor. Opcional.
\end{itemize}

\par Estos costes pueden variar considerablemente dependiendo de la plataforma sobre la que se despliegue la solución.
\par La opción más económica será aquella en la que la plataforma hardware y los técnicos disponibles pueden asumir la carga adicional que la solución requiere; en este escenario sólo se deberá contemplar el coste del
soporte técnico, el cual es así mismo opcional. Por lo tanto en este escenario la solución no tendría un {\bf coste económico directo}.

\par Otro posible escenario, consistiría en desplegar la solución en algún proveedor del Cloud. Esta arquitectura es cada vez más frecuente y es importante evaluar su coste.
\par En primer lugar se debe estimar el coste de la plataforma en sí. Para ello se ha utilizado la herramienta online proporcionada por Azure~\cite{azurecalculator}.


\begin{figure}[!ht]
  \centering
  \begin{tabular}{|p{0.15\textwidth}|p{0.13\textwidth}|p{0.35\textwidth}|p{0.2\textwidth}|}
  \hline
  {\bf Service type}              & {\bf Region}  & {\bf Description}                                                                           & {\bf Estimated Cost}                \\
  \hline
  Azure Kubernetes Service (AKS)  & West Europe   & 5 D2 v3 (2 vCPU(s), 8 GB RAM) nodes x 730 Hours; Pay as you go; 1 managed OS disks – S4     & 370.66€                             \\
  \hline
  Storage Accounts                & West Europe   & File Storage, Standard Performance Tier, General Purpose V2, LRS Redundancy, 100 GB \
                          Capacity, 1 Put or Create Container operations, 1 List operations, 1 Other operations, 0 Additional Sync servers      & 5.09€                               \\
  \hline
  Container Registry              & West Europe   & Basic Tier, 1 units x 30 days, 100 GB Bandwidth                                             & 11.18€                              \\
  \hline
  Support                         &               & Support                                                                                     & 0.00€                               \\
  \hline
                                  &               & Licensing Program                                                                           & Microsoft Online Services Agreement \\
  \hline
                                  &               & Monthly Total                                                                               & 386.93€                             \\
  \hline
                                  &               & Annual Total                                                                                & 4643.18€                            \\
  \hline
  \end{tabular}
  \label{tabla:azurecalculatoraks}
  \caption{Estimación económica de la plataforma de contenedores en Microsoft Azure AKS~\cite{aks} (fuente~\cite{azurecalculator})}
\end{figure}

\par A este importe hay que añadir el tiempo en horas del administrador de la plataforma. Según la misma plataforma de Azure, esta plataforma requiere una media de quince horas y media anuales con un coste de cincuenta
euros por hora. Lo que nos da la siguiente estimación:

\begin{figure}[!ht]
  \centering
  \begin{tabular}{|p{0.3\textwidth}|p{0.15\textwidth}|p{0.15\textwidth}|p{0.2\textwidth}|}
  \hline
  {\bf Recurso}               & {\bf Tiempo}  & {\bf Coste/hora}    & {\bf Coste Total}           \\
  \hline
    Administrador DevOps      & 15.5 horas  & 50€                 & 775€ \\
  \hline
                              &             & Total anual         & 775€ \\
  \hline
  \end{tabular}
  \label{tabla:azurecalculatorsalary}
  \caption{Estimación económica de Microsoft Azure (fuente~\cite{azurecalculator})}
\end{figure}

\par Por último, para calcular el coste de soporte técnico, dado que la solución está compuesta de múltiples tecnologías, no se considera viable añadir el coste del soporte técnico de todas ellas.
\par A modo de referencia se ha evaluado el coste del soporte 24/7 de Nginx para cuatro instancias y el coste es el siguiente:

\begin{figure}[!ht]
  \centering
  \begin{tabular}{| c | c | c | c |}
  \hline
  {\bf Tipo de soporte}           & {\bf Número de instancias}    & {\bf Coste/instancia}   & {\bf Coste Total}   \\
  \hline
    Professional (soporte 24/7)   & 4                             & 3109.02 / año           & 12436.06€           \\
  \hline
                                  &                               & Total anual             & 12436.06€           \\
  \hline
  \end{tabular}
  \label{tabla:nginxplus}
  \caption{Coste del soporte técnico de Nginx (fuente~\cite{nginxplus})}
\end{figure}

\par Se pueden ver los costes consolidados en la \hyperref[tabla:costesresumen]{Tabla resumen de la viabilidad económica} \\
\begin{figure}[!ht]
  \centering
  \begin{tabular}{| p{0.37\textwidth} | c | r | r | r |}
  \hline
    {\bf Entorno}                                              & {\bf Plataforma}  & {\bf Administración}  & {\bf Soporte} & {\bf Coste total}   \\
  \hline
    Entorno Cloud con soporte Nginx                            & 4643€                   & 775€                        & 12436.06 €           & 17854.06 €         \\
  \hline
    Entorno Cloud sin soporte Nginx                            & 4643€                   & 775€                        & 0 €                  & 5418 €             \\
  \hline
    Entorno on-premise sin HW adicional con soporte Nginx      & 0€                      & 775€                        & 12436.06 €           & 13211.06 €         \\
  \hline
    Entorno on-premise sin HW adicional sin soporte Nginx      & 0€                      & 775€                        & 0 €                  & 775 €              \\
  \hline
  \end{tabular}
  \label{tabla:costesresumen}
  \caption{Resumen de costes estimados}
\end{figure}

\par Es importante reseñar que se ha considerado el soporte Nginx como un caso representativo del elevado coste que tienen este tipo de servicios. Habitualmente no es necesario
contratar este tipo de soluciones en entornos que no sean críticos o muy sensibles y, en estos entornos, se suele requerir un soporte equivalente de todos los componentes.
\par En los entornos en los que no se requiere un servicio de este tipo, es habitual optar por servicios profesionales y de consultoría, sea bajo demanda o con la contratación de
una bolsa de horas.
