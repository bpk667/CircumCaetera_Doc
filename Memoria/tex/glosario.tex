\newacronym{gcd}{GCD}{Greatest Common Divisor}
\newacronym{lcm}{LCM}{Least Common Multiple}
\newacronym{waf}{WAF}{Web Application Firewall}
\newacronym{saas}{SaaS}{Software as a Service}
\newacronym{cdn}{CDN}{Content Delivery Network}
\newacronym{dos}{DoS}{Denial-of-service}
\newacronym{SNI}{SNI}{Server Name Indication\cite{TLSSNI}}
\newacronym{ev}{EV}{Extended Validation}
\newacronym{rbac}{RBAC}{role-based access control}
\newacronym{dbf}{DBF}{Database Firewall}
\newacronym{rgdp}{RGDP}{Reglamento General de Protección de Datos}
\newacronym{pcidss}{PCI DSS}{Payment Card Industry Data Security Standard}
\newacronym{xss}{XSS}{Cross-site scripting\cite[Artículo en OWASP]{owaspxss}}
\newacronym{sqli}{SQLi}{SQL injection\cite[Artículo en OWASP]{owaspsqli}}
\newacronym{poc}{PoC}{Proof of concept}
\newacronym{gpl}{GPL}{GNU General Public License\cite[Licencia GPL]{gpl}}
\newacronym{lgpl}{LGPL}{GNU Lesser General Public License\cite[Licencia LGPL]{lgpl}}
\newacronym{siem}{SIEM}{Security information and event management}
\newacronym{ca}{CA}{Certification Authority}


\newglossaryentry{throughput}
{
    name=Throughput,
    description={La tasa de transferencia efectiva (en inglés throughput) es el volumen de trabajo o de información neto que fluye a través de
    un sistema, como puede ser una red de computadoras.}~\cite[Wikipedia]{wikithroughput}
}

\newglossaryentry{CA}
{
    name=CA,
    description={En criptografía, las expresiones autoridad de certificación, o certificadora, o certificante, o las siglas AC o CA (por la
    denominación en idioma inglés Certification Authority), señalan a una entidad de confianza, responsable de emitir y revocar los certificados,
    utilizando en ellos la firma electrónica, para lo cual se emplea la criptografía de clave pública.}~\cite[Wikipedia]{CA}
}

\newglossaryentry{Atacante}
{
    name=Atacante,
    description={El atacante es un individuo u organización que intenta obtener el control de un sistema informático para utilizarlo con fines
    maliciosos, robo de información o de hacer daño a su objetivo.}~\cite[Wikipedia]{Atacante}
}

