\chapter{Conclusiones}
\par Se considera que se ha conseguido el objetivo principal de la solución: Crear una herramienta de software libre que permita mejorar la
seguridad en las plataformas web en entornos en los que no se dispone de los medios necesarios.

\par La solución que se ha construido implanta un WAF libre con las reglas recomendadas por OWASP \cite{owaspcrs}, un frontal TLS con una configuración que sigue las mejores
prácticas de la industria \cite{TLSBestPractices} y una serie de controles de seguridad adicionales como son las cabeceras HTTP, el control de cookies o HTTP/2.

\par Estos y otros controles se encuentran disponibles de forma gratuita si el administrador de la plataforma web tiene el conocimiento y tiempo para configurar su plataforma
adecuadamente, pero esta actividad requiere lidiar con dependencias de software, mantenerse actualizado en materia de seguridad y añade complejidad al entorno. O,
alternativamente, se puede invertir en una solución privativa que cubra algunas de las funcionalidades y que tiene un coste económico considerable.

\par Con esta solución se espera unir lo mejor de ambas opciones: Facilidad de despliegue y mantenimiento con un coste ajustado que puedan permitirse plataformas con medios
limitados.

\par Adicionalmente, se garantiza la adaptabilidad de la solución, debido por un lado a la licencia  \acrlong{gpl} y por otro al bajo solapamiento entre los distintos componentes,
lo que facilitará su uso en distintos entornos en los que las necesidades sean dispares.

\par No obstante, si el entorno es crítico o se requiere un soporte 24/7, el coste de soporte es inevitable en el futuro cercano.

\par El reto que se plantea una vez entregada esta primera versión es empezar a desplegar la solución en entornos heterogéneos para aprender y solucionar los potenciales errores o
limitaciones que puedan surgir tal como se verá con mayor detalle en la siguiente sección.

\section{Trabajo a futuro}
\par Como en cualquier solución de software existen dos actividades sobre las que se deben trabajar: Por un lado se debe consolidar y
estabilizar la plataforma para garantizar su funcionamiento y, por otro lado, se deben implementar nuevas funcionalidades.
\subsection{Consolidar y estabilizar la solución}
\par Para mejorar la estabilidad de la plataforma y depurar posibles errores, se procederá a implantar la solución en varias plataformas web
de empresas con las que colaboro y de las que ya he conseguido un acuerdo.
\par Algunas de estas plataformas tienen un tráfico considerable y cientos de dominios, lo que permitirá realizar pruebas de humo en
profundidad y se espera que la solución sea suficientemente madura una vez esté desplegada en producción en estos entornos.

\subsection{Nuevas funcionalidades}
\par Dado que los ataques y los controles de seguridad están en continua evolución, se espera que se vayan añadiendo controles de seguridad
de for continua.
\par Adicionalmente, se han identificado una serie de hitos que permitirán mejorar la adopción de la solución una vez ésta esté disponible
públicamente:
\begin{itemize}
  \item Definir ficheros de configuración que faciliten los despliegues en entornos más complejos.
  \item Definir un modo Simulación / Bloqueo que permita activar y desactivar los mecanismos de seguridad de forma sencilla.
  \item Implementar controles adicionales: {\em Rate limit}, bots autorizados, control de reputación, etc.
  \item Implementar pruebas automáticas: pruebas funcionales, pruebas de seguridad, etc.
\end{itemize}

