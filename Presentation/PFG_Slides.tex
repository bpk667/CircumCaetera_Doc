\documentclass[xcolor=table]{beamer}
\usepackage[spanish]{babel}
\usepackage[utf8]{inputenc}
\usepackage{listings}
\usepackage[table]{xcolor}
\usepackage{multicol}
\usepackage{colortbl}
\usepackage{amssymb}
\usepackage{pifont}
\usepackage[autostyle]{csquotes} 
\usepackage[toc,acronyms]{glossaries}
\usepackage[
  backend=biber,
  style=numeric,
  sorting=none
]{biblatex}

\newcommand{\autor}{Pedro Pozuelo Rodríguez}
\newcommand{\titulo}{Reverse Proxy con capacidades de\\Firewall de aplicación web y aceleración TLS}
\newcommand{\titulocorto}{Reverse Proxy + WAF + aceleración TLS}
\newcommand{\profesor}{Ana del Valle Corrales Paredes}
\newcommand{\PFGID}{Pendiente}
\newcommand{\fechaversion}{\today}

\newcommand*\greenarrow{%
  \color{green!80!black}{$\Uparrow$}
}

\newcommand*\redarrow{%
  \color{red!80!black}{$\Downarrow$}
}

\newcommand*\yellowarrow{%
  \color{yellow!80!black}{$\Rightarrow$}
}

\mode<presentation> {
  \usetheme{Warsaw}
  %\beamerdefaultoverlayspecification{<+->} %Every new bullet includes \pause
}
% \setbeamersize{text margin left=1cm, text margin right=1cm}

\title[\titulocorto]{\titulo}
\author[\autor]{Alumno: \autor \and \and \and \and \and \and \and \and \and \and \and% Really crappy approach for getting authors in different lines:(
Directora: \profesor}
\institute{Universidad Europea\\Proyecto de Fin de Grado}
\date{\fechaversion}
\logo{\includegraphics[height=1cm]{fig/universidad_logo}}

% Remove navigation icons
\beamertemplatenavigationsymbolsempty

% Bibliography and glossaries.
\addbibresource{bib/Referencias.bib}
\makeglossaries
\newacronym{http}{HTTP}{Hypertext Transfer Protocol}
\newacronym{w3c}{W3C}{World Wide Web Consortium}
\newacronym{csp}{CSP}{Content-Security-Policy}
\newacronym{https}{HTTPS}{Hypertext Transfer Protocol Secure}
\newacronym{hsts}{HSTS}{HTTP Strict Transport Security}
\newacronym{mitm}{MitM}{Man-in-the-middle}
\newacronym{url}{URL}{Uniform Resource Locator}
\newacronym{dlp}{DLP}{Data loss prevention}
\newacronym{utm}{UTM}{Unified Threat Management}
\newacronym{ssl}{SSL}{Secure Sockets Layer}
\newacronym{tls}{TLS}{Transport Layer Security}
\newacronym{nids}{NIDS}{Network Intrusion Detection System}
\newacronym{gcd}{GCD}{Greatest Common Divisor}
\newacronym{lcm}{LCM}{Least Common Multiple}
\newacronym{waf}{WAF}{Web Application Firewall}
\newacronym{saas}{SaaS}{Software as a Service}
\newacronym{cdn}{CDN}{Content Delivery Network}
\newacronym{dos}{DoS}{Denial-of-service}
\newacronym{SNI}{SNI}{Server Name Indication\cite[Wikipedia]{wiki:TLSSNI}}
\newacronym{sdlc}{SDLC}{Systems Development Life Cycle\cite[Wikipedia]{wiki:SDLC}}
\newacronym{ev}{EV}{Extended Validation}
\newacronym{rbac}{RBAC}{role-based access control}
\newacronym{dbf}{DBF}{Database Firewall}
\newacronym{rgdp}{RGDP}{Reglamento General de Protección de Datos}
\newacronym{pcidss}{PCI DSS}{Payment Card Industry Data Security Standard}
\newacronym{owasp}{OWASP}{Open Web Application Security Project}
\newacronym{xss}{XSS}{Cross-site scripting\cite{owaspxss}}
\newacronym{sqli}{SQLi}{Inyección SQL\cite{owaspsqli}}
\newacronym{xsrf}{XSRF}{Cross-site request forgery}
\newacronym{poc}{PoC}{Proof of concept}
\newacronym{gpl}{GPL}{GNU General Public License\cite[Licencia GPL]{gpl}}
\newacronym{lgpl}{LGPL}{GNU Lesser General Public License\cite[Licencia LGPL]{lgpl}}
\newacronym{siem}{SIEM}{Security information and event management}
\newacronym{ca}{CA}{Certification Authority}
\newacronym{acme}{ACME}{Automatic Certificate Management Environment\cite[Estándar ACME]{rfc8555}}
\newacronym{cicd}{CI/CD}{Continuous Integration and Continuous Deployment or Continuous Delivery \cite[What is CI/CD]{cicd}}

\newglossaryentry{DefensaProfundidad}
{
    name={Defensa en Profundidad},
    description={El concepto de defensa en profundidad se basa en la premisa de que todo componente de un sistema puede ser vulnerado, y por tanto no se debe delegar la seguridad
    de un sistema en un único método o componente de protección.}~\cite[Wikipedia]{wiki:DefensaProfundidad}
}

\newglossaryentry{cloud}
{
    name={La nube},
    description={La computación en la nube (del inglés cloud computing), conocida también como servicios en la nube, informática en la nube, nube de cómputo, nube de conceptos o simplemente «la nube», es un paradigma que permite ofrecer
    servicios de computación a través de una red, que usualmente es Internet.}~\cite[Wikipedia]{wiki:cloud}
}

\newglossaryentry{throughput}
{
    name=Throughput,
    description={La tasa de transferencia efectiva (en inglés throughput) es el volumen de trabajo o de información neto que fluye a través de
    un sistema, como puede ser una red de computadoras.}~\cite[Wikipedia]{wiki:throughput}
}

\newglossaryentry{CA}
{
    name=CA,
    description={En criptografía, las expresiones autoridad de certificación, o certificadora, o certificante, o las siglas AC o CA (por la
    denominación en idioma inglés Certification Authority), señalan a una entidad de confianza, responsable de emitir y revocar los certificados,
    utilizando en ellos la firma electrónica, para lo cual se emplea la criptografía de clave pública.~\cite[Wikipedia]{wiki:ca}}
}

\newglossaryentry{Atacante}
{
    name=Atacante,
    description={El atacante es un individuo u organización que intenta obtener el control de un sistema informático para utilizarlo con fines
    maliciosos, robo de información o de hacer daño a su objetivo.~\cite[Wikipedia]{wiki:atacante}}
}

\longnewglossaryentry{ManifiestoAgil}
{
    name={Manifiesto por el Desarrollo Ágil de Software},
  }
{
    \par Estamos descubriendo formas mejores de desarrollar software tanto por nuestra propia experiencia como ayudando a terceros. A través de este trabajo hemos aprendido a valorar:
    \par {\Large Individuos e interacciones} sobre procesos y herramientas\\
    {\Large Software funcionando} sobre documentación extensiva\\
    {\Large Colaboración con el cliente} sobre negociación contractual\\
    {\Large Respuesta ante el cambio} sobre seguir un plan
    \par Esto es, aunque valoramos los elementos de la derecha, valoramos más los de la izquierda.~\cite{ManifiestoAgil}
}


\newglossaryentry{Clickjacking}
{
    name=Clickjacking,
    description={es una técnica maliciosa para engañar a usuarios de Internet con el fin de que revelen información confidencial o tomar control de su ordenador cuando hacen clic en
    páginas web aparentemente inocentes.~\cite[Wikipedia]{wiki:Clickjacking}}
}

\newglossaryentry{MitM}
{
    name=MitM,
    description={es un ataque en el que se adquiere la capacidad de leer, insertar y modificar a voluntad. El atacante debe ser capaz de observar e interceptar mensajes entre las dos víctimas y procurar que ninguna de las
    víctimas conozca que el enlace entre ellos ha sido violado.~\cite[Wikipedia]{wiki:mitm}}.
}

\newglossaryentry{XSS}
{
    name=\acrlong{xss},
    description={es un tipo de vulnerabilidad informática o agujero de seguridad típico de las aplicaciones Web, que puede permitir a una tercera persona inyectar en páginas web visitadas por el usuario código JavaScript o
    en otro lenguaje similar (ej: VBScript).~\cite[Wikipedia]{wiki:xss}}.
}

\newglossaryentry{Phishing}
{
    name=Phishing,
    description={denomina un modelo de abuso informático y que se comete mediante el uso de un tipo de ingeniería social, caracterizado por intentar adquirir información confidencial de forma
    fraudulenta~\cite[Wikipedia]{wiki:Phishing}}.
}

\newglossaryentry{XSRF}
{
    name=\acrlong{xsrf},
    description={es un tipo de exploit malicioso de un sitio web en el que comandos no autorizados son transmitidos por un usuario en el cual el sitio web confía.~\cite[Wikipedia]{wiki:XSRF}}.
}

\newglossaryentry{SQLI}
{
    name=\acrlong{sqli},
    description={es un método de infiltración de código intruso que se vale de una vulnerabilidad informática presente en una aplicación en el nivel de validación de las entradas para realizar operaciones sobre una base de
    datos~\cite[Wikipedia]{wiki:sqli}}
}

\newglossaryentry{Cookie}
{
    name=\acrlong{Cookie},
    description={es una pequeña información enviada por un sitio web y almacenada en el navegador del usuario, de manera que el sitio web puede consultar la actividad previa del navegador~\cite[Wikipedia]{wiki:cookie}}
}


\glstocfalse

\begin{document}
\AtBeginSection[] {
  \begin{frame}[shrink=20]
    \frametitle{\titulocorto}
    \tableofcontents[currentsection]
  \end{frame}
}

\begin{frame}
  \titlepage
\end{frame}

% Logo for the Titlepage
\logo{\includegraphics[height=0.5cm]{fig/universidad_logo_notext}}

% Logo for all the other pages
\titlegraphic{\includegraphics[width=2cm]{universidad_logo}\hspace*{4.75cm}~%
  \includegraphics[width=2cm]{universidad_logo}
}

\begin{frame}[shrink=20]
  \frametitle{Agenda}
  \begin{itemize}
    \item Introducción:
      \begin{itemize}
        \item Aplicaciones web y la seguridad.
        \item Mecanismos de protección.
        \item ¿Qué es un WAF?
      \end{itemize}
    \item Situación actual. Estado del arte:
      \begin{itemize}
        \item Soluciones WAF privativas.
        \item Soluciones WAF de software libre.
        \item Comparativa soluciones actuales.
      \end{itemize}
    \item Solución.
      \begin{itemize}
        \item Objetivo.
        \item Diseño.
        \item Arquitectura.
      \end{itemize}
    \item Test y resultados.
    \item Conclusiones.
  \end{itemize}
\end{frame}

\section{Introducción}
\subsection{Aplicaciones web y la seguridad}
\begin{frame}[shrink=20]
  \frametitle{Aplicaciones web y la seguridad}
  Algunos ejemplos de uso de protocolos \acrshort{http} y \acrshort{https}:
  \begin{itemize}
    \item Aplicaciones web.
    \item Aplicaciones móviles.
    \item {\em Internet of things} (IoT):edificios inteligentes, {\em wearables}, etc.
    \item Arquitectura de microservicios.
    \item \acrlong{doh} (\acrshort{doh} \cite{doh}).
  \end{itemize}
  \begin{block}{HTTP + TLS}
    HTTPS es cada vez más utilizado en todo tipo de aplicaciones y no se limita a las aplicaciones web como venía siendo tradicionalmente.
  \end{block}
\end{frame}

\begin{frame}[shrink=10]
  \frametitle{Aplicaciones web y la seguridad}
  \begin{block}{Premisa}
     La seguridad 100\% no existe.
  \end{block}
  Las aplicaciones web están siendo atacadas continuamente.
  \begin{figure}
    \includegraphics[width=0.35\textwidth]{fig/application-attacks-2}%
    \qquad \includegraphics[width=0.35\textwidth]{fig/Vulnerabilidades_OWASP}
    \caption{\small{Ataques en capa de aplicación (fuentes Arbor \cite{articleArbor}, \acrshort{owasp} \cite{owasptop10})}}
    \label{fig:applicationattacks}
  \end{figure}
  \begin{alertblock}{Conclusión}
      Se debe realizar un esfuerzo continuo para mejor la seguridad de las plataformas web.
  \end{alertblock}
\end{frame}

\begin{frame}[shrink=20]
  \frametitle{Vulnerabilidades recientes en canales cifrados}
  Otro componente en el que se han descubierto múltiples vulnerabilidades críticas son los canales SSL/TLS.
  \begin{center}
    \rowcolors[]{1}{blue!20}{blue!5}
    \begin{tabular}{|l|l|}
      \hline
      {\bf Vulnerabilidad}   			& {\bf Componente afectado}\\
      \hline
      POODLE											&   SSL ver. 3.0        \\
      \hline
      BEAST												&   TLS ver. 1.0        \\
      \hline
      CRIME                       &   TLS compression     \\
      \hline
      BREACH                      &   HTTP compression    \\
      \hline
      Heartbleed                  &   OpenSSL ver. 1.0.1  \\
      \hline
    \end{tabular}
  \end{center}
  \begin{block}{Solución - Mitigación}
    Actualizar software y desactivar las versiones o el componente afectados.
    \par El riesgo de afectar la funcionalidad de la plataforma es bajo (dependiendo del entorno).
  \end{block}
\end{frame}

\subsection{Mecanismos de protección}
\begin{frame}[shrink]
  \frametitle{Soluciones, controles de seguridad}
  \begin{itemize}
    \item {\bf Desarrollo de código seguro}: metodologías y herramientas.
      \par Retos: Tiempo y recursos.
    \item {\bf Aplicar un ciclo de vida de aplicaciones}: Gestión de actualizaciones y configuración segura.
      \par Retos: Una actualización puede afectar al entorno, el objetivo es que la {\em aplicación funcione}.
      \par {\em chmod 777} o {\em iptables -A INPUT -j ACCEPT} funcionan.
    \item {\bf Protección perimetral de red}: Firewall de red, Sistema de Prevención de Intrusos.
      \par Reto: Visibilidad reducida.
    \item {\bf Herramientas de firewall de aplicación.}: \acrshort{waf}.
      \par Reto: Elevado coste o complejo de mantener.
  \end{itemize}
\end{frame}

\begin{frame}[shrink=20]
  \frametitle{Soluciones. Estándares y protocolos}
  Existen múltiples iniciativas cuyo objetivo es mejorar la seguridad de las aplicaciones web:
  \begin{itemize}
    \item Metodología del Ciclo de Vida de Desarrollo de Software (\acrshort{sdlc} del inglés).
    \item Estándares como el {\em Payment Card Industry Data Security Standard} (PCI DSS~\cite{pcidssrequirements}).
    \item TLS versión 1.3.
    \item HTTP/2.
    \item TLS Server Name Indication (SNI~\cite{wiki:TLSSNI}).
    \item Security Headers.
  \end{itemize}
  \begin{block}{Uso e implementación}
    Estas soluciones están disponibles y ofrecen mecanismos válidos para mejorar la seguridad de las plataformas web pero su implementación puede ser compleja.
  \end{block}
\end{frame}

\begin{frame}[shrink=10]
  \frametitle{Uso e implementación}
  Las alternativas implican un coste elevado, implementar soluciones complejas o aceptar el riesgo de seguridad. Y el resultado es el siguiente:
	\begin{columns}
	\begin{column}{0.3\textwidth}
		\begin{figure}
			\includegraphics[width=0.9\textwidth]{fig/ImplementationHTTPHTTPS}
      \caption{\small{Tráfico HTTP versus HTTPS ~\cite{thesslstore}}}
		\end{figure}
	\end{column}
	\begin{column}{0.7\textwidth}  %%<--- here
		\begin{figure}
			\includegraphics[width=0.7\textwidth]{fig/ImplementationSSLTLS}
      \caption{\small{Máxima versión SSL/TLS soportada~\cite{thesslstore}}}
		\end{figure}
	\end{column}
	\end{columns}
  \begin{block}{Uso e implementación}
  Se ha elegido la versión SSL/TLS como ejemplo de un vector de ataque conocido popularmente cuya mitigación es sencilla.
  \end{block}
\end{frame}

\begin{frame}[shrink=20]
  \frametitle{Visibilidad reducida}
    \begin{figure}[c]
			\includegraphics[width=0.9\textwidth]{fig/TLS_Visibility}
      \caption{\small{Evolución de HTTP a HTTPS con {\em Diffie–Hellman}}}
		\end{figure}
\end{frame}

\subsection{¿Qué es un WAF?}
\begin{frame}[shrink=20,fragile]
  \frametitle{\acrlong{waf} (\acrshort{waf})}
  \begin{exampleblock}{¿Qué es un WAF?}
    Un WAF es una herramienta especializada en filtrar, monitorizar y bloquear las conexiones desde y hacia una aplicación web (Fuente: Instituto Nacional de Ciberseguridad~\cite{incibewaf}).
  \end{exampleblock}
  Características principales:
  \begin{itemize}
    \item Analiza el tráfico web: Entiende GET, POST, parámetros URL, etc.
    \item Se aplican políticas y reglas de filtrado. Por ejemplo:
        \begin{lstlisting}
admin'--
' or 1=1#
' or 1=1-- -
        \end{lstlisting}
    \item Listas blancas o negras de User Agents, IP, caracteres aceptados en URL o formularios, etc.
  \end{itemize}
\end{frame}


\section{Estado del arte}


\subsection{Soluciones WAF privativas}
\begin{frame}[shrink]
  \frametitle{Soluciones WAF privativas}
  \begin{itemize}
    \item Tecnología 1.
    \item Tecnología 1.
    \item Tecnología 1.
  \end{itemize}
\end{frame}

\begin{frame}[shrink]
  \frametitle{Ventajas e inconvenientes}
  PRO: Facilidad de despliegue y gestión.
  PRO: Funcionalidades adicionales (p.e. CDN).
  CONS: Coste.
  CONS: Falta de flexibilidad.
\end{frame}

\subsection{Soluciones WAF de software libre}
\begin{frame}[shrink]
  \frametitle{Soluciones WAF de software libre}
  \begin{itemize}
    \item Tecnología 1.
    \item Tecnología 1.
    \item Tecnología 1.
  \end{itemize}
\end{frame}

\begin{frame}[shrink]
  \frametitle{Ventajas e inconvenientes}
  PRO: Coste.
  CONS: Complejidad.
\end{frame}

\subsection{Comparativa soluciones actuales}
\begin{frame}[shrink]
  \frametitle{Comparativa soluciones actuales}
  TODO
\end{frame}

\section{Solucion}
\subsection{Objetivo}
\begin{frame}[shrink]
  \frametitle{Objetivo}
  Como respuesta a la situación actual, se define el siguiente objetivo:
  \begin{block}{Objetivo}
    Construir una solución de software libre con capacidades de WAF y aceleración SSL/TLS, que sea  fácilmente desplegable y que minimice el esfuerzo y el impacto que dicha
    solución tiene sobre la plataforma web actual o futura.
    \par También debe ser fácilmente adaptable a diferentes necesidades y entornos.
  \end{block}
\end{frame}

\subsection{Diseño}
\begin{frame}[shrink]
  \frametitle{Diseño}
  \begin{figure}
    \includegraphics[width=0.9\textwidth]{fig/Diagram_HLD}
    \caption{\small{Diseño a alto nivel de la solución}}
  \end{figure}
\end{frame}

\begin{frame}[shrink]
  \frametitle{Componentes del WAF}
  \begin{figure}
    \includegraphics[width=0.9\textwidth]{fig/Diagram_HLD}
    \caption{\small{Diseño a alto nivel de la solución}}
  \end{figure}
\end{frame}

\begin{frame}[shrink]
  \frametitle{Componentes}
  \par Componentes de la solución:
  \begin{tabular}{ l c }
      \acrlong{waf}                                               & \parbox[c]{5em}{\includegraphics[width=.30\textwidth,height=3cm,keepaspectratio]{fig/ModSecurityLogo}} \\
      Software criptográfico                                      & \parbox[c]{5em}{\includegraphics[width=.30\textwidth,height=3cm,keepaspectratio]{OpenSSL_logo}} \\
      virtualización (contenedores)                               & \parbox[c]{5em}{\includegraphics[width=.30\textwidth,height=3cm,keepaspectratio]{fig/DockerLogo}} \\
      Automatización y orquestación.                              & \parbox[c]{5em}{\includegraphics[width=.30\textwidth,height=3cm,keepaspectratio]{fig/KubernetesLogo}} \\
      Gestión de certificados.                                    & \parbox[c]{5em}{\includegraphics[width=.30\textwidth,height=3cm,keepaspectratio]{fig/LetsEncryptLogo}} \\
      Políticas y controles de seguridad.                         & \parbox[c]{5em}{\includegraphics[width=.30\textwidth,height=3cm,keepaspectratio]{fig/OWASPCRSLogo}} \\
    \end{tabular}
\end{frame}


\subsection{Arquitectura}
\begin{frame}[shrink]
  \frametitle{Arquitectura. Gestión de certificados}
  \begin{figure}
    \includegraphics[width=0.9\textwidth]{fig/Diagram_HTTP_Services}
  \end{figure}
\end{frame}

\begin{frame}[shrink]
  \frametitle{Arquitectura. Peticiones HTTP/HTTPS}
  \begin{figure}
    \includegraphics[width=0.9\textwidth]{fig/Diagram_HTTP_Services}
  \end{figure}
\end{frame}


\section{Conclusiones}
\begin{frame}[shrink]
  \frametitle{Conclusiones}
  \par Características de la propuesta
  \begin{itemize}
    \item \greencheck Independencia de la plataforma web    
    \item \greencheck Independencia operacional (RBAC)      
    \item \yellowcheck Complejidad (despliegue y operación) 
    \item \greencheck Coste económico                       
    \item \redcrossed Soporte técnico                       
    \item \greencheck Información accesible por terceros    
    \item \greencheck Adaptabilidad / Personalización       
    \item \greencheck Acceso al código fuente               
    \item \greencheck Funcionalidades adicionales           
  \end{itemize}
\end{frame}

\begin{frame}[shrink]
  \frametitle{Conclusiones}
  \begin{itemize}
    \item Se ha conseguido crear una solución de {\bf software libre} que permita mejorar la seguridad en las plataformas web en
      entornos sin los medios o conocimientos necesarios. \\
    \item Mejores prácticas de OWASP \cite{owaspbestpractices} y Qualys \cite{TLSBestPractices}. \\
    \item Implementa la facilidad de las soluciones privativas. \\
    \item Adaptabilidad del software libre. \\
    \item Funcionalidades adicionales: Cabeceras de seguridad, cookies, etc. \\
  \end{itemize}
\end{frame}


\begin{frame}[shrink]
  \frametitle{Trabajo a futuro}
  \begin{itemize}
    \item Desplegar en producción. \\
    \item Consolidar y estabilizar la solución. \\
    \item Añadir funcionalidades: \\
      \begin{itemize}
        \item Anti-DDoS (rate limit). \\
        \item Control de bots. \\
        \item Mejorar los ficheros de configuración \\
        \item Botón de modo simulación / modo bloqueo. \\
        \item Botón de modo depuración. \\
    \end{itemize}
  \end{itemize}
\end{frame}



\section{Tests y resultados}
\begin{frame}[shrink]
  \frametitle{Tests y resultados}
  TODO
\end{frame}



\begin{frame}
  \begin{center}
    \frametitle{Ruegos y preguntas}
    \huge ¿Preguntas?
  \end{center}
\end{frame} 

\appendix
\begin{frame}[allowframebreaks]
	\frametitle{Referencias}
	\printbibliography[title=Referencias]
\end{frame}

\begin{frame}[allowframebreaks]
	\frametitle{Glosario}
	\printglossaries
\end{frame}

\end{document}
