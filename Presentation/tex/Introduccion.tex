\theoremstyle{remark} \newtheorem*{NICDomain}{NIC Domain}
\theoremstyle{remark} \newtheorem*{Site}{Site}
\theoremstyle{remark} \newtheorem*{IPDB}{Internet Protocol Databse (IPDB)}
\theoremstyle{remark} \newtheorem*{Node}{Node}
\theoremstyle{remark} \newtheorem*{DS}{Database Server (DS, D-SRV)}
\theoremstyle{remark} \newtheorem*{AS}{Application Server (AS, A-SRV)}
\theoremstyle{remark} \newtheorem*{LC}{Local Collector (LC)}
\theoremstyle{remark} \newtheorem*{RC}{Remote Collector (RC)}

\section{Introducción}
\begin{frame}[shrink]
  \frametitle{Agenda}
  \begin{itemize}
    \item Introducción:
      \begin{itemize}
        \item Aplicaciones web y la seguridad.
        \item Qué es un Web Application Firewall (WAF).
        \item Comunicaciones cifradas. Transport Layer Security (TLS).
      \end{itemize}
    \item Situación actual. Estado del arte:
      \begin{itemize}
        \item Soluciones WAF privativas.
        \item Soluciones WAF de software libre.
        \item Uso de HTTP / HTTPS.
      \end{itemize}
    \item Solución.
      \begin{itemize}
        \item Objetivo.
        \item Diseño.
        \item Arquitectura.
      \end{itemize}
    \item Conclusiones.
    \item Test y resultados.
  \end{itemize}
\end{frame}

% \begin{frame}
%   \frametitle{Situación actual}
%   \begin{itemize}
%     \item 
%     \item Applicaciones
%       \begin{itemize}
%         \item Event Viewer (Módulo {\em Analysis})
%           \begin{itemize}
%             \item Muestra los eventos en un estado de pre-proceso.
%             \item Eventos tal y como son almacenados en la IPDB.
%           \end{itemize}
%         \item Query (Módulo {\em Analysis})
%           \begin{itemize}
%             \item Eventos procesados y categorizados en diferentes tablas.
%             \item Variables asignadas a campos.
%             \item Se dejan de ver ``todos los eventos''.
%           \end{itemize}
%         \item Reports (Módulo {\em Reports})
%       \end{itemize}
%   \end{itemize}
% \end{frame}
% 
% 
% \begin{frame}[shrink]
%   \frametitle{Conceptos RSA enVision}
%   \begin{Site}
%    Un cluster de servidores de la serie LS conectados a un servidor de almacenamiento de datos ({\em NAS}).
%   \end{Site}
% 
%   \begin{NICDomain}
%   Dos o más Sites interconectados. Se denomina {\em NIC} si existe un único {\em NIC Domain}.
%   \end{NICDomain}
%    
% \begin{IPDB}
%   Todos los eventos pertenecientes a un NIC Domain. Se accede a través de 1 o más Data Servers.
% \end{IPDB}
% 
%   \begin{Node}
%     Es un servidor que pertenece a un Site.
%   \end{Node}
% \end{frame}
% 
% \begin{frame}[shrink]
%   \frametitle{Apliances RSA enVision}
%   \begin{DS}
%       Se encarga de las comunicaciones entre los demás servidores pertenecientes al Site y se comunica con los demás Sites.\\
%       Controla las peticiones a la IPDB. \\
%       Recibe las peticiones del Application Server y realiza las peticiones a Local Collectors, Remote Collectors y otros Database Servers.
%   \end{DS}
% 
%   \begin{AS}
%       Appliance encargado de analizar los datos.
%       Se conecta al DS y proporciona acceso a la Interfaz web de enVision (GUI del cliente).
%   \end{AS}
% 
%   \begin{LC}
%       Recolecta y almacena los eventos. \\
%       Mantiene una comunicación permanente con el DS.
%   \end{LC}
% 
%   \begin{RC}
%       Recolecta y almacena los eventos y los reenvía a los Sites remotos.
%       Es un tipo especial de colector que periódicamente envía los datos al master DS.
%       Contiene un DS propio, por lo que es un Site independiente.
%   \end{RC}
% \end{frame}
% 
% \begin{frame}
%   \frametitle{Arquitectura RSA enVision}
%   \begin{figure}[htb]
%     \centering
%     \includegraphics[width=0.8\textwidth]
%       {Introduccion/img/enVision_Functional.png}
%   \end{figure}
% \end{frame}
% 
% \begin{frame}[plain]
%   \frametitle{Arquitectura RSA enVision}
%   \begin{figure}[htb]
%     \centering
%     \includegraphics[width=1\textwidth]
%       {Introduccion/img/enVision_Architecture.pdf}
%   \end{figure}
% \end{frame}
% 
% \begin{frame}
%   \frametitle{Diseño enVision ONO}
%   \begin{figure}[htb]
%     \centering
%     \includegraphics[width=0.7\textwidth]
%       {Introduccion/img/enVision_Functional-ONO.png}
%   \end{figure}
% \end{frame}
% 
% \begin{frame}[plain]
%   \frametitle{Arquitectura RSA enVision ONO}
%   \begin{figure}[htb]
%     \centering
%     \includegraphics[width=0.8\textwidth]
%       {Introduccion/img/enVision_ONO.png}
%   \end{figure}
% \end{frame}
% \section{Acceso a la información}
% \begin{frame}
%   \frametitle{Acceso a la información}
%   \begin{itemize}
%     \item Analisis de los datos históricos y tareas forenses.
%     \item Applicaciones
%       \begin{itemize}
%         \item Event Viewer (Módulo {\em Analysis})
%           \begin{itemize}
%             \item Muestra los eventos en un estado de pre-proceso.
%             \item Eventos tal y como son almacenados en la IPDB.
%           \end{itemize}
%         \item Query (Módulo {\em Analysis})
%           \begin{itemize}
%             \item Eventos procesados y categorizados en diferentes tablas.
%             \item Variables asignadas a campos.
%             \item Se dejan de ver ``todos los eventos''.
%           \end{itemize}
%         \item Reports (Módulo {\em Reports})
%       \end{itemize}
%   \end{itemize}
% \end{frame}
% 
% \setbeamercovered{dynamic}
% \subsection{Event Viewer}
% \begin{frame}
%   \frametitle{Event Viewer}
% 
%   \only<1-2>{
%   \begin{figure}[htb]
%     \centering
%     \includegraphics[width=0.5\textwidth]
%       {Introduccion/img/User_EventViewer.png}
%   \end{figure}}
% 
%   \onslide<2->
%   Permite mostrar los mensajes:
%   \begin{itemize}
%     \onslide<2,3>\item En tiempo real, especificar un rango temporal y filtrar por tipo de dispositivo y tipo de evento.
%     \onslide<2,4>\item Filtrar por nivel de severidad o por cadena de texto en las opciones avanzadas. 
%     \onslide<2,5>\item Mostrar gráficas de los eventos.
%   \end{itemize}
% 
%   \begin{center}
%       \only<3>{\includegraphics[width=0.8\textwidth]{Introduccion/img/ScreenShot_EventViewer.png}}
%       \only<4>{\includegraphics[width=0.9\textwidth]{Introduccion/img/ScreenShot_EventViewer_Adv.png}}
%       \only<5>{\includegraphics[width=0.6\textwidth]{Introduccion/img/ScreenShot_EventViewer_Graph.png}}
%   \end{center}
% \end{frame}
% 
% \setbeamercovered{invisible}
% \subsection{Queries / Consultas}
% \begin{frame}
%   \frametitle{Queries / Consultas}
%   \begin{center}
%     \includegraphics[width=0.5\textwidth]{Introduccion/img/User_Queries.png}
%   \end{center}
% 
%   \pause
%   \begin{itemize}[<+->]
%     \item Permite analizar los contenidos de una tabla de la base de datos de manera rápida.
%     \item Se pueden configurar filtros y ejecutarse queries de manera ad-hoc.
%     \item Se pueden almacenar queries personalizadas.
%     \item Los resultados pueden exportarse a un fichero {\em .csv}. El cual puede abrirse y manipularse desde aplicaciones tipo Excel.
%     \item Se utiliza para definir informes y gráficos.
%     \item Se utiliza para localizar y resolver problemas con informes y gráficos.
%   \end{itemize}
% \end{frame}
% 
% \begin{frame}
%   \frametitle{GUI: Queries / Consultas}
%   \begin{center}
%       \includegraphics[width=0.9\textwidth]{Introduccion/img/ScreenShot_Queries.png}
%   \end{center}
% \end{frame}
% 
% \begin{frame}
%   \frametitle{Sintaxis: Queries / Consultas}
%   \begin{itemize}
%     \item Para encontrar una coincidencia, se introduce la cadena concreta.
%     \item Soporta LIKE y carácteres especiales para comparar cadenas:
%       \begin{itemize}
%         \item ``\%'' representa cualquier grupo de carácteres (0 o más).
%         \item ``\_'' representa un caracter cualquiera.
%       \end{itemize}
%     \item Los filtros relacionados horizontalmente se unen mediante un AND lógico.
%     \item Los filtros relacionados verticalmente se unen mediante un OR lógico.
%   \end{itemize}
% \end{frame}
% 
% \subsection{Reports / Informes}
% \begin{frame}[shrink]
%   \frametitle{Reports / Informes}
%   \begin{itemize}[<+->]
%     \item En cuanto al formato, hay 2 tipos de informes:
%       \begin{itemize}
%         \item {\em Tabular}: Tienen formato de tabla y está compuesto por filas y columnas.
%         \item {\em Graph}: Son de tipo gráfico y a su vez pueden ser de tipo bar, line o pie
%       \end{itemize}
%     \item Se pueden ejecutar de manera puntual ({\em ad-hoc}) o de manera programada ({\em scheduled})
%     \item El resultado se puede dar en formato {\em html}, {\em csv} o {\em pdf}.
%     \item Hay una gran variadad de informes predefinidos.
%     \item {\em Custom Report}: Informes personalizado.
%     \item {\em Bind reports}: Es un metainforme que agrupa la ejecución de varios informes para agilizar su ejecución y minimizar la carga del sistema.
%     \item {\em Reports folder}: Carpeta en la que se pueden agrupar los informes programados.
%   \end{itemize}
% \end{frame}
% 
% \begin{frame}
%   \frametitle{Report Groups}
%   Dentro de los informes que viene de manera predefinida en enVision, éstos viene categorizados según las siguientes categorías de dispositivos ({\em Device class}):
%   \begin{itemize}
%     \item Compliance
%     \item Correlated Alerts
%     \item Host Devices
%     \item Network Devices
%     \item Security Devices
%     \item Storage Devices
%     \item Task Triage
%     \item VAM
%   \end{itemize}
% \end{frame}
% 
% \subsubsection{Ad Hoc Reports}
% \begin{frame}
%   \frametitle{Ad Hoc Reports}
%   \begin{itemize}
%     \item Se pueden ejecutar en cualquier momento.
%     \item Diseñados para ejecutarse inmediatamente.
%     \item Pueden tardar bastante tiempo en generarse, dependiendo de la cantidad de datos a procesar y los recursos disponibles.
%     \item Puede programarse su ejecución en un momento concreto.
%   \end{itemize}
% \end{frame}
% 
% \subsubsection{Bind Reports}
% \begin{frame}
%   \frametitle{Bind Reports}
%   \begin{itemize}
%     \item Agrupa la ejecución de varios informes para agilizar su ejecución y minimizar la carga del sistema.
%     \item Son útiles cuando se agrupan informes que acceden a la misma información.
%   \end{itemize} \pause
%   \begin{center}
%       \includegraphics[width=0.9\textwidth]{Introduccion/img/ScreenShot_BindReport.png}
%   \end{center}
% \end{frame}
% 
% \begin{frame}
%   \frametitle{Bind Reports}
%   \begin{itemize}
%     \item Para mejorar el rendimiento, marcar la opción {\em Reuse the data source when possible}
%   \end{itemize} \pause
%   \begin{center}
%       \includegraphics[width=0.9\textwidth]{Introduccion/img/ScreenShot_BindReport_Reuse.png}
%   \end{center}
% \end{frame}
% 
% \subsubsection{Schedulled Reports - Informes programados}
% \begin{frame}
%   \frametitle{Schedulled Reports - Informes programados}
%   Se utilizan para generar informes de tipo {\em Tabular}, {\em Graph} o {\em Bind}.
%   \begin{itemize}
%     \item Útiles para acceder a múltiples informes.
%     \item Útiles para ejecutar informes en momentos de mínima carga de los sistemas.
%   \end{itemize} \pause
%   \begin{center}
%       \includegraphics[width=0.9\textwidth]{Introduccion/img/ScreenShot_SchedulledReports.png}
%   \end{center}
% \end{frame}
% 
% \begin{frame}
%   \frametitle{Schedulled Reports: Set Recurrence}
%   Se define la recurrencia con la que se ejecutarán los informes
%   \begin{center}
%       \includegraphics[width=0.9\textwidth]{Introduccion/img/ScreenShot_SchedulledReports_Recurrence.png}
%   \end{center}
% \end{frame}
% 
% \begin{frame}
%   \frametitle{Schedulled Reports: Report Directory}
%   \begin{itemize}
%     \item {\em Report Directory}: Indica el directorio donde se guardan los resultados de los informes.
%     \item Se pueden almacenar en formato {\em csv} y {\em pdf}
%   \end{itemize} \pause
%   \begin{center}
%       \includegraphics[width=0.9\textwidth]{Introduccion/img/ScreenShot_ReportDirectory.png}
%   \end{center}
% \end{frame}
% 
% \section{Personalización de informes}
% \begin{frame}
%   \frametitle{Personalización de informes Ad Hoc}
%   La personalización de informes permite:
%   \begin{itemize}
%     \item Modificar informes ya existentes, ya sean de tipo tabla {\em Tabular} o gráfico {\em Graph}.
%     \item Crear informes nuevos utilizando informes pervios como plantilla.
%     \item Crear informes nuevos desde cero.
%   \end{itemize}
%   Los elementos a definir en el proceso son:
%   \begin{itemize}
%     \item Seleccionar de la tabla donde están alojados los datos.
%     \item Seleccionar los campos que se van a presentar en el informe.
%     \item Definir las cláusura SQL y las variables.
%     \item Modificar la presentación del informe.
%   \end{itemize}
% \end{frame}
% 
% \begin{frame}
%   \frametitle{Ad Hoc Reports: Configuración}
%       \begin{figure}[htb]
%         \centering
%         \includegraphics[width=0.7\textwidth]
%           {Introduccion/img/ScreenShot_AdHoc_Report_Name.png}
%           \caption{Definimos el nombre del informe}
%       \end{figure}
% \end{frame}
% 
% \begin{frame}
%   \frametitle{Ad Hoc Reports: Configuración, continuación}
%       \begin{figure}[htb]
%         \centering
%         \includegraphics[width=0.7\textwidth]
%         {Introduccion/img/ScreenShot_AdHoc_Report_Table.png}
%         \caption{Seleccionamos la tabla y los campos sobre los que queremos configurar el informe}
%       \end{figure}
% \end{frame}
% 
% \begin{frame}
%   \frametitle{Ad Hoc Reports: Configuración, continuación}
%       \begin{figure}[htb]
%         \centering
%         \includegraphics[width=0.7\textwidth]
%         {Introduccion/img/ScreenShot_AdHoc_Report_Order.png}
%         \caption{Definimos los campos de ordenación}
%       \end{figure}
% \end{frame}
% 
% \begin{frame}
%   \frametitle{Ad Hoc Reports: Configuración, continuación}
%       \begin{figure}[htb]
%         \centering
%         \includegraphics[width=0.7\textwidth]
%         {Introduccion/img/ScreenShot_AdHoc_Report_Where.png}
%         \caption{Definimos los parámetros que se pedirán en momento de ejecución y los filtros de la cláusula WHERE}
%       \end{figure}
% \end{frame}
% 
% \begin{frame}
%   \frametitle{Ad Hoc Reports: Configuración, continuación}
%       \begin{figure}[htb]
%         \centering
%         \includegraphics[width=0.7\textwidth]
%         {Introduccion/img/ScreenShot_AdHoc_Report_Param.png}
%         \caption{Definición de parámetros}
%       \end{figure}
% \end{frame}
% 
% \begin{frame}
%   \frametitle{Ad Hoc Reports: Configuración, continuación}
%       \begin{figure}[htb]
%         \centering
%         \includegraphics[width=0.7\textwidth]
%           {Introduccion/img/ScreenShot_AdHoc_Report_SQLWhere.png}
%           \caption{Definición de la clausula WHERE}
%       \end{figure}
% \end{frame}
% 
% \defverbatim[colored]\stringcode{%
%   \begin{lstlisting}[frame=single,emph={ga},emphstyle=\color{olive}]
% DeviceAddress = '123.123.1.1'
% ForeignAddress = '192.168.%'
%   \end{lstlisting}}
% 
% \defverbatim[colored]\datecode{%
%   \begin{lstlisting}[frame=single,emph={ga},emphstyle=\color{olive}]
% Date/Time >= '10-24-06 9:00'
% AND 
% Date/Time <= '10-24-06 13:00'
%   \end{lstlisting}}
% 
% \begin{frame}
%   \frametitle{Sentencias SQL}
%   \begin{itemize}[<+->]
%     \item Cadenas de texto dentro de una sentencia SQL: Pueden ser una dirección IP, un fecha, hora...
%       \begin{itemize}
%         \item Acotadas por comillas simples
%         \item Son sensibles a las mayúsculas/minúsculas (case-sensitive)
%         \item Se pueden utilizar operadores lógicos {\em AND}, {\em OR} y {\em NOT}. \\
%           Nota: Los valores que son números no se consideran cadenas de texto. (p.e. un puerto).
%       \end{itemize}
%     \item {\em DeviceAddress} e {\em IP Address} son cadenas. \\
%       Ejemplo de uso:
%         \stringcode
%   \end{itemize}
% \end{frame}
% 
% \begin{frame}
%   \frametitle{Sentencias SQL, continuación}
%   \begin{itemize}
%     \item Formato de fechas:
%       \begin{itemize}
%         \item Acotadas por comillas simples
%         \item Los formatos que enVision admite son: \\
%           mm-dd-yy \\
%           mm-dd-yyyy \\
%           mm/dd/yy \\
%           mm/dd/yyyy \\
%           Month dd, yy \\
%           Month dd, yyyy 
%          \item Ejemplo de uso:
%             \datecode
%       \end{itemize}
%   \end{itemize}
% \end{frame}
% 
% \begin{frame}
%   \frametitle{Global Tables}
%   \begin{itemize}
%     \item Los informes y las consultas basados en una Global Table muestran la información de múltiples dispositivos en un único informe.
%     \item Algunas de estas tablas son:
%       \begin{itemize}
%         \item {\em Global Table}: Agrupan los datos de múltiples dispositivos con los campos cuyo uso es más frecuente.
%         \item {\em Global Summary table}: Agrupa todos los datos de las tablas de tipo {\em summary}
%         \item {\em Baseline Table}: Contiene información del número de mensajes recibidos por {\em Alert category}, {\em Alert level} y {\em NIC Category}.
%         \item {Alerts table, Baseline table, Syslog Rate table, Accounting Address Summary table, Access Control Accounting table, Alert Trends table}...
%       \end{itemize}
%   \end{itemize}
% \end{frame}
