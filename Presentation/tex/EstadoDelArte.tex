\section{Estado del arte}


\subsection{Soluciones WAF privativas}
\begin{frame}[shrink]
  \frametitle{Soluciones WAF privativas}
  Destacan las siguientes soluciones:
  \begin{itemize}
    \item {\bf Soluciones WAF SaaS}. Desplegados en las instalaciones del fabricante - o el Cloud - y gestionados por el mismo.
      \begin{itemize}
        \item {\em Cloud Web Application Firewall} \cite{cloudflarewaf} de Cloudflare\cite{cloudflare}.
        \item {\em Kona WAF\cite{kona}} de {\em Akamai\cite{akamai}}.
        \item {\em Incapsula\cite{Incapsula}}.
      \end{itemize}
    \item {\bf Soluciones WAF tipo appliance o máquina virtual}. Máquinas o instancias dedicadas en las que se tiene un acceso exclusivamente a la configuración de la aplicación.
      \begin{itemize}
        \item {\em Imperva WAF Gateway\cite{imperva}}.
        \item {\em Fortiweb\cite{fortiweb}} de la empresa {\em Fortinet\cite{fortinet}}.
      \end{itemize}
  \end{itemize}
\end{frame}

\begin{frame}[shrink]
  \frametitle{Soluciones WAF privativas SaaS}
  Las soluciones WAF \acrshort{saas} ofrecen una serie de funcionalidades adicionales:
  \begin{itemize}
    \item Red de distribución de contenidos (\acrlong{cdn} del inglés).
    \item Protección contra ataques de denegación de servicio (\acrshort{dos} del inglés) en capa de aplicación.
    \item Caché de contenido estático.
    \item Suscripción a listas de reputación de IP, dominios o Localizador de recursos uniforme (\acrshort{url} del inglés).
    \item Bloqueo de bots maliciosos.
    \item Sistema de creación de informes.
  \end{itemize}
\end{frame}

\begin{frame}[shrink]
  \frametitle{Soluciones WAF privativas tipo appliance}
  Las soluciones WAF de tipo appliance ofrecen a su vez las siguientes funcionalidades adicionales:
  \begin{itemize}
    \item Crear perfiles de las aplicaciones web y filtrar parámetros no permitidos.
    \item Parcheo virtual de vulnerabilidades mediante la integración con escaneadores de vulnerabilidades.
    \item Suscripción a listas de reputación de IP, dominios o URL.
    \item Aceleración TLS.
    \item Bloqueo de bots maliciosos.
    \item Sistema de creación de informes.
    \item Antivirus.
  \end{itemize}
\end{frame}

\subsection{Soluciones WAF de software libre}
\begin{frame}[shrink]
  \frametitle{Soluciones WAF de software libre}
  Existen múltiples soluciones de software libre
  \begin{itemize}
    \item IronBee\cite{IronBee}.
    \item WebCastellum\cite{WebCastellum}.
    \item RAPTOR\cite{raptor}.
    \item NAXSI\cite{NAXSI}.
    \item OpenWAF\cite{openwaf}.
    \item FreeWAF\cite{freewaf}.
    \item Shadow Daemon\cite{ShadowDaemon}.
    \item AQTRONiX WebKnight\cite{WebKnight}.
    \item Vulture\cite{vulture}.
    \item ModSecurity \cite{modsecurity}.
  \end{itemize}
  \begin{block}{ModSecurity}
  \par Entre ellas destaca ModSecurity por ser la solución de software libre más extendida y activa de la comunidad e implementa un número significativo de los controles de seguridad deseables en un WAF.
  \end{block}
\end{frame}

\begin{frame}[shrink]
  \frametitle{Ventajas e inconvenientes}
  {\bf Ventajas}:
  \begin{itemize}
    \item Más económicos.
    \item Acceso al código fuente y la capacidad de modificarlo.
    \item (en la mayoría de las soluciones) elimina la dependencia del proveedor.
    \item Más adaptables a las necesidades de cada entorno.
  \end{itemize}
  {\bf Desventajas}:
  \begin{itemize}
    \item Dependencia de la plataforma web (tradicionalmente un módulo de ésta).
    \item Más difíciles de implementar y de mantener.
    \item Proceso de depuración de errores es más complejo.
    \item Actualización o migración de la plataforma web y el WAF deben realizarse conjuntamente.
  \end{itemize}
\end{frame}

\subsection{Comparativa soluciones actuales}
\begin{frame}[shrink]
  \frametitle{Comparativa soluciones actuales}
  Privativas SaaS:
  PRO: Facilidad de despliegue y gestión.
  PRO: Funcionalidades adicionales (p.e. CDN).
  CONS: Coste.
  CONS: Falta de flexibilidad.

  Software libre:
  PRO: Coste, flexibilidad, adaptabilidad.
  CONS: Complejidad.
\end{frame}

