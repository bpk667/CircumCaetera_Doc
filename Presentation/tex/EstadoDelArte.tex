\section{Estado del arte}

\subsection{Soluciones WAF privativas}
\begin{frame}[shrink]
  \frametitle{Soluciones WAF privativas}
  Destacan las siguientes soluciones:
  \begin{itemize}
    \item {\bf Soluciones WAF SaaS}. Desplegados en las instalaciones del fabricante - o el Cloud - y gestionados por el mismo.
      \begin{itemize}
        \item {\em Cloud Web Application Firewall} \cite{cloudflarewaf} de Cloudflare\cite{cloudflare}.
        \item {\em Kona WAF\cite{kona}} de {\em Akamai\cite{akamai}}.
        \item {\em Incapsula\cite{Incapsula}}.
      \end{itemize}
    \item {\bf Soluciones WAF tipo appliance o máquina virtual}. Máquinas o instancias dedicadas en las que se tiene un acceso exclusivamente a la configuración de la aplicación.
      \begin{itemize}
        \item {\em Imperva WAF Gateway\cite{imperva}}.
        \item {\em Fortiweb\cite{fortiweb}} de la empresa {\em Fortinet\cite{fortinet}}.
      \end{itemize}
  \end{itemize}
\end{frame}

\begin{frame}[shrink]
  \frametitle{Ventajas e inconvenientes}
  {\bf Ventajas}:
  \begin{itemize}
    \item Sencillas de implementar
    \item Independencia de la infraestructura de la plataforma web.
    \item (\acrshort{rbac} del inglés).
    \item Soporte técnico.
    \item Funcionalidades adicionales.
  \end{itemize}
  {\bf Desventajas}:
  \begin{itemize}
    \item Elevado coste económico.
    \item No es posible adaptar la solución a necesidades específicas.
  \end{itemize}
\end{frame}

\subsection{Soluciones WAF de software libre}
\begin{frame}[shrink]
  \frametitle{Soluciones WAF de software libre}
  Existen múltiples soluciones de software libre
  \begin{itemize}
    \item IronBee\cite{IronBee}.
    \item WebCastellum\cite{WebCastellum}.
    \item RAPTOR\cite{raptor}.
    \item NAXSI\cite{NAXSI}.
    \item OpenWAF\cite{openwaf}.
    \item FreeWAF\cite{freewaf}.
    \item Shadow Daemon\cite{ShadowDaemon}.
    \item AQTRONiX WebKnight\cite{WebKnight}.
    \item Vulture\cite{vulture}.
    \item ModSecurity \cite{modsecurity}.
  \end{itemize}
  \begin{block}{ModSecurity}
  \par Entre ellas destaca ModSecurity por ser la solución de software libre más extendida y activa de la comunidad e implementa un número significativo de los controles de seguridad deseables en un WAF.
  \end{block}
\end{frame}

\begin{frame}[shrink]
  \frametitle{Ventajas e inconvenientes}
  {\bf Ventajas}:
  \begin{itemize}
    \item Más económicos.
    \item Acceso al código fuente y la capacidad de modificarlo.
    \item (en la mayoría de las soluciones) elimina la dependencia del proveedor.
    \item Más adaptables a las necesidades de cada entorno.
  \end{itemize}
  {\bf Desventajas}:
  \begin{itemize}
    \item Dependencia de la plataforma web (tradicionalmente un módulo de ésta).
    \item Más difíciles de implementar y de mantener.
    \item Proceso de depuración de errores es más complejo.
    \item Actualización o migración de la plataforma web y el WAF deben realizarse conjuntamente.
  \end{itemize}
\end{frame}


\subsection{Comparativa soluciones WAF}

\begin{frame}[shrink,c]
  \frametitle{Comparativa soluciones}
  A continuación se muestra un resumen de las características de las distintas soluciones y la solución del presente proyecto:
  \begin{center}
  \rowcolors[]{1}{blue!20}{blue!5}
  \resizebox{\linewidth}{!} {
  \begin{tabular}{| l | c | c | c | c |}
    \hline
    {\bf Características}                       &	{\bf WAF SaaS}          & {\bf WAF Appliance} & {\bf WAF Software libre}  & {\bf Propuesta}       \\
    \hline
    Independencia de la plataforma web  & \textcolor{green}{Muy buena}& \textcolor{green}{Buena}& \textcolor{red}{Mala}& \textcolor{green}{Buena}\\
    \hline
    Independencia operacional (RBAC)    & \textcolor{green}{Muy buena}& \textcolor{green}{Buena}& \textcolor{red}{Mala}& \textcolor{green}{Buena}\\
    \hline
    Complejidad (despliegue y operación)& \textcolor{green}{Muy baja}& \textcolor{green}{Baja}& \textcolor{red}{Alta}& \textcolor{yellow}{Media}\\
    \hline
    Coste económico                     & \textcolor{red}{Alto}& \textcolor{red}{Alto}& \textcolor{green}{Bajo}& \textcolor{green}{Bajo}\\
    \hline
    Soporte técnico                     & \textcolor{green}{Bueno}& \textcolor{green}{Bueno}& \textcolor{red}{Limitado}& \textcolor{red}{Limitado}\\
    \hline
    Información accesible por terceros  & \textcolor{red}{Sí}& \textcolor{green}{No}& \textcolor{green}{No}& \textcolor{green}{No}\\ 
    \hline
    Adaptabilidad / Personalización     & \textcolor{red}{Muy baja}& \textcolor{red}{Baja}& \textcolor{green}{Alta}& \textcolor{green}{Alta}\\ 
    \hline
    Acceso al código fuente             & \textcolor{red}{No}& \textcolor{red}{No}& \textcolor{green}{Sí}   & \textcolor{green}{Sí}   \\ 
    \hline
    Funcionalidades adicionales  & \textcolor{green}{Muy buenas}& \textcolor{green}{Buenas}& \textcolor{red}{Limitadas}& \textcolor{green}{Buenas}\\
    \hline
  \end{tabular}}
  \end{center}
\end{frame}

