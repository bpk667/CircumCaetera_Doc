\newacronym{http}{HTTP}{Hypertext Transfer Protocol}
\newacronym{https}{HTTPS}{Hypertext Transfer Protocol Secure}
\newacronym{hsts}{HSTS}{HTTP Strict Transport Security\cite[Wikipedia]{wiki:hsts}}
\newacronym{url}{URL}{Uniform Resource Locator}
\newacronym{dlp}{DLP}{Data loss prevention}
\newacronym{utm}{UTM}{Unified Threat Management}
\newacronym{ssl}{SSL}{Secure Sockets Layer}
\newacronym{tls}{TLS}{Transport Layer Security}
\newacronym{nids}{NIDS}{Network Intrusion Detection System}
\newacronym{gcd}{GCD}{Greatest Common Divisor}
\newacronym{lcm}{LCM}{Least Common Multiple}
\newacronym{waf}{WAF}{Web Application Firewall}
\newacronym{saas}{SaaS}{Software as a Service}
\newacronym{cdn}{CDN}{Content Delivery Network}
\newacronym{dos}{DoS}{Denial-of-service}
\newacronym{SNI}{SNI}{Server Name Indication\cite[Wikipedia]{wiki:TLSSNI}}
\newacronym{sdlc}{SDLC}{Systems Development Life Cycle\cite[Wikipedia]{wiki:SDLC}}
\newacronym{ev}{EV}{Extended Validation}
\newacronym{rbac}{RBAC}{role-based access control}
\newacronym{dbf}{DBF}{Database Firewall}
\newacronym{rgdp}{RGDP}{Reglamento General de Protección de Datos}
\newacronym{pcidss}{PCI DSS}{Payment Card Industry Data Security Standard}
\newacronym{owasp}{OWASP}{Open Web Application Security Project}
\newacronym{xss}{XSS}{Cross-site scripting\cite[Artículo en OWASP]{owaspxss}}
\newacronym{sqli}{SQLi}{SQL injection\cite[Artículo en OWASP]{owaspsqli}}
\newacronym{poc}{PoC}{Proof of concept}
\newacronym{gpl}{GPL}{GNU General Public License\cite[Licencia GPL]{gpl}}
\newacronym{lgpl}{LGPL}{GNU Lesser General Public License\cite[Licencia LGPL]{lgpl}}
\newacronym{siem}{SIEM}{Security information and event management}
\newacronym{ca}{CA}{Certification Authority}
\newacronym{acme}{ACME}{Automatic Certificate Management Environment\cite[Estándar ACME]{rfc8555}}
\newacronym{cicd}{CI/CD}{Continuous Integration and Continuous Deployment or Continuous Delivery \cite[What is CI/CD]{cicd}}

\newglossaryentry{DefensaProfundidad}
{
    name={Defensa en Profundidad},
    description={El concepto de defensa en profundidad se basa en la premisa de que todo componente de un sistema puede ser vulnerado, y por tanto no se debe delegar la seguridad
    de un sistema en un único método o componente de protección.}~\cite[Wikipedia]{wiki:DefensaProfundidad}
}

\newglossaryentry{cloud}
{
    name={La nube},
    description={La computación en la nube (del inglés cloud computing), conocida también como servicios en la nube, informática en la nube, nube de cómputo, nube de conceptos o simplemente «la nube», es un paradigma que permite ofrecer
    servicios de computación a través de una red, que usualmente es Internet.}~\cite[Wikipedia]{wiki:cloud}
}

\newglossaryentry{throughput}
{
    name=Throughput,
    description={La tasa de transferencia efectiva (en inglés throughput) es el volumen de trabajo o de información neto que fluye a través de
    un sistema, como puede ser una red de computadoras.}~\cite[Wikipedia]{wiki:throughput}
}

\newglossaryentry{CA}
{
    name=CA,
    description={En criptografía, las expresiones autoridad de certificación, o certificadora, o certificante, o las siglas AC o CA (por la
    denominación en idioma inglés Certification Authority), señalan a una entidad de confianza, responsable de emitir y revocar los certificados,
    utilizando en ellos la firma electrónica, para lo cual se emplea la criptografía de clave pública.~\cite[Wikipedia]{wiki:ca}}
}

\newglossaryentry{Atacante}
{
    name=Atacante,
    description={El atacante es un individuo u organización que intenta obtener el control de un sistema informático para utilizarlo con fines
    maliciosos, robo de información o de hacer daño a su objetivo.~\cite[Wikipedia]{wiki:atacante}}
}

\longnewglossaryentry{ManifiestoAgil}
{
    name={Manifiesto por el Desarrollo Ágil de Software},
  }
{
    \par Estamos descubriendo formas mejores de desarrollar software tanto por nuestra propia experiencia como ayudando a terceros. A través de este trabajo hemos aprendido a valorar:
    \par {\Large Individuos e interacciones} sobre procesos y herramientas\\
    {\Large Software funcionando} sobre documentación extensiva\\
    {\Large Colaboración con el cliente} sobre negociación contractual\\
    {\Large Respuesta ante el cambio} sobre seguir un plan
    \par Esto es, aunque valoramos los elementos de la derecha, valoramos más los de la izquierda.~\cite{ManifiestoAgil}
}


